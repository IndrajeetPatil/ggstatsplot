\documentclass[]{article}
\usepackage{lmodern}
\usepackage{amssymb,amsmath}
\usepackage{ifxetex,ifluatex}
\usepackage{fixltx2e} % provides \textsubscript
\ifnum 0\ifxetex 1\fi\ifluatex 1\fi=0 % if pdftex
  \usepackage[T1]{fontenc}
  \usepackage[utf8]{inputenc}
\else % if luatex or xelatex
  \ifxetex
    \usepackage{mathspec}
  \else
    \usepackage{fontspec}
  \fi
  \defaultfontfeatures{Ligatures=TeX,Scale=MatchLowercase}
\fi
% use upquote if available, for straight quotes in verbatim environments
\IfFileExists{upquote.sty}{\usepackage{upquote}}{}
% use microtype if available
\IfFileExists{microtype.sty}{%
\usepackage{microtype}
\UseMicrotypeSet[protrusion]{basicmath} % disable protrusion for tt fonts
}{}
\usepackage[margin=1in]{geometry}
\usepackage{hyperref}
\PassOptionsToPackage{usenames,dvipsnames}{color} % color is loaded by hyperref
\hypersetup{unicode=true,
            pdftitle={ggstatsplot: ggplot2 Based Plots with Statistical Details},
            pdfauthor={Indrajeet Patil; Mina Cikara},
            colorlinks=true,
            linkcolor=blue,
            citecolor=Blue,
            urlcolor=Blue,
            breaklinks=true}
\urlstyle{same}  % don't use monospace font for urls
\usepackage{color}
\usepackage{fancyvrb}
\newcommand{\VerbBar}{|}
\newcommand{\VERB}{\Verb[commandchars=\\\{\}]}
\DefineVerbatimEnvironment{Highlighting}{Verbatim}{commandchars=\\\{\}}
% Add ',fontsize=\small' for more characters per line
\usepackage{framed}
\definecolor{shadecolor}{RGB}{248,248,248}
\newenvironment{Shaded}{\begin{snugshade}}{\end{snugshade}}
\newcommand{\AlertTok}[1]{\textcolor[rgb]{0.94,0.16,0.16}{#1}}
\newcommand{\AnnotationTok}[1]{\textcolor[rgb]{0.56,0.35,0.01}{\textbf{\textit{#1}}}}
\newcommand{\AttributeTok}[1]{\textcolor[rgb]{0.77,0.63,0.00}{#1}}
\newcommand{\BaseNTok}[1]{\textcolor[rgb]{0.00,0.00,0.81}{#1}}
\newcommand{\BuiltInTok}[1]{#1}
\newcommand{\CharTok}[1]{\textcolor[rgb]{0.31,0.60,0.02}{#1}}
\newcommand{\CommentTok}[1]{\textcolor[rgb]{0.56,0.35,0.01}{\textit{#1}}}
\newcommand{\CommentVarTok}[1]{\textcolor[rgb]{0.56,0.35,0.01}{\textbf{\textit{#1}}}}
\newcommand{\ConstantTok}[1]{\textcolor[rgb]{0.00,0.00,0.00}{#1}}
\newcommand{\ControlFlowTok}[1]{\textcolor[rgb]{0.13,0.29,0.53}{\textbf{#1}}}
\newcommand{\DataTypeTok}[1]{\textcolor[rgb]{0.13,0.29,0.53}{#1}}
\newcommand{\DecValTok}[1]{\textcolor[rgb]{0.00,0.00,0.81}{#1}}
\newcommand{\DocumentationTok}[1]{\textcolor[rgb]{0.56,0.35,0.01}{\textbf{\textit{#1}}}}
\newcommand{\ErrorTok}[1]{\textcolor[rgb]{0.64,0.00,0.00}{\textbf{#1}}}
\newcommand{\ExtensionTok}[1]{#1}
\newcommand{\FloatTok}[1]{\textcolor[rgb]{0.00,0.00,0.81}{#1}}
\newcommand{\FunctionTok}[1]{\textcolor[rgb]{0.00,0.00,0.00}{#1}}
\newcommand{\ImportTok}[1]{#1}
\newcommand{\InformationTok}[1]{\textcolor[rgb]{0.56,0.35,0.01}{\textbf{\textit{#1}}}}
\newcommand{\KeywordTok}[1]{\textcolor[rgb]{0.13,0.29,0.53}{\textbf{#1}}}
\newcommand{\NormalTok}[1]{#1}
\newcommand{\OperatorTok}[1]{\textcolor[rgb]{0.81,0.36,0.00}{\textbf{#1}}}
\newcommand{\OtherTok}[1]{\textcolor[rgb]{0.56,0.35,0.01}{#1}}
\newcommand{\PreprocessorTok}[1]{\textcolor[rgb]{0.56,0.35,0.01}{\textit{#1}}}
\newcommand{\RegionMarkerTok}[1]{#1}
\newcommand{\SpecialCharTok}[1]{\textcolor[rgb]{0.00,0.00,0.00}{#1}}
\newcommand{\SpecialStringTok}[1]{\textcolor[rgb]{0.31,0.60,0.02}{#1}}
\newcommand{\StringTok}[1]{\textcolor[rgb]{0.31,0.60,0.02}{#1}}
\newcommand{\VariableTok}[1]{\textcolor[rgb]{0.00,0.00,0.00}{#1}}
\newcommand{\VerbatimStringTok}[1]{\textcolor[rgb]{0.31,0.60,0.02}{#1}}
\newcommand{\WarningTok}[1]{\textcolor[rgb]{0.56,0.35,0.01}{\textbf{\textit{#1}}}}
\usepackage{longtable,booktabs}
\usepackage{graphicx,grffile}
\makeatletter
\def\maxwidth{\ifdim\Gin@nat@width>\linewidth\linewidth\else\Gin@nat@width\fi}
\def\maxheight{\ifdim\Gin@nat@height>\textheight\textheight\else\Gin@nat@height\fi}
\makeatother
% Scale images if necessary, so that they will not overflow the page
% margins by default, and it is still possible to overwrite the defaults
% using explicit options in \includegraphics[width, height, ...]{}
\setkeys{Gin}{width=\maxwidth,height=\maxheight,keepaspectratio}
\IfFileExists{parskip.sty}{%
\usepackage{parskip}
}{% else
\setlength{\parindent}{0pt}
\setlength{\parskip}{6pt plus 2pt minus 1pt}
}
\setlength{\emergencystretch}{3em}  % prevent overfull lines
\providecommand{\tightlist}{%
  \setlength{\itemsep}{0pt}\setlength{\parskip}{0pt}}
\setcounter{secnumdepth}{5}
% Redefines (sub)paragraphs to behave more like sections
\ifx\paragraph\undefined\else
\let\oldparagraph\paragraph
\renewcommand{\paragraph}[1]{\oldparagraph{#1}\mbox{}}
\fi
\ifx\subparagraph\undefined\else
\let\oldsubparagraph\subparagraph
\renewcommand{\subparagraph}[1]{\oldsubparagraph{#1}\mbox{}}
\fi

%%% Use protect on footnotes to avoid problems with footnotes in titles
\let\rmarkdownfootnote\footnote%
\def\footnote{\protect\rmarkdownfootnote}

%%% Change title format to be more compact
\usepackage{titling}

% Create subtitle command for use in maketitle
\providecommand{\subtitle}[1]{
  \posttitle{
    \begin{center}\large#1\end{center}
    }
}

\setlength{\droptitle}{-2em}

  \title{ggstatsplot: ggplot2 Based Plots with Statistical Details}
    \pretitle{\vspace{\droptitle}\centering\huge}
  \posttitle{\par}
    \author{Indrajeet Patil\footnote{Harvard University, \href{mailto:patilindrajeet.science@gmail.com}{\nolinkurl{patilindrajeet.science@gmail.com}}} \\ Mina Cikara\footnote{Harvard University}}
    \preauthor{\centering\large\emph}
  \postauthor{\par}
      \predate{\centering\large\emph}
  \postdate{\par}
    \date{2019-08-13}

\usepackage{amsmath}
\usepackage{amssymb}
\usepackage{longtable}
\usepackage{booktabs}
\usepackage{amsfonts}
\usepackage{hyperref}
\usepackage{booktabs}
\usepackage{float}
\usepackage[dvipsnames]{xcolor}
\usepackage{fontawesome}

\begin{document}
\maketitle

{
\hypersetup{linkcolor=black}
\setcounter{tocdepth}{2}
\tableofcontents
}
\begin{quote}
``What is to be sought in designs for the display of information is the clear
portrayal of complexity. Not the complication of the simple; rather \ldots{} the
revelation of the complex.''\\
- Edward R. Tufte
\end{quote}

\hypertarget{introduction-raison-detre}{%
\section{Introduction: Raison d'être}\label{introduction-raison-detre}}

\href{https://indrajeetpatil.github.io/ggstatsplot/}{\texttt{ggstatsplot}} is an extension
of \href{https://github.com/tidyverse/ggplot2}{\texttt{ggplot2}} package for creating
graphics with details from statistical tests included in the plots themselves
and targeted primarily at behavioral sciences community to provide a one-line
code to produce information-rich plots. In a typical exploratory data analysis
workflow, data visualization and statistical modeling are two different phases:
visualization informs modeling, and modeling in its turn can suggest a
different visualization method, and so on and so forth. The central idea of
\texttt{ggstatsplot} is simple: combine these two phases into one in the form of
graphics with statistical details, which makes data exploration simpler and
faster.

\hypertarget{need-for-informative-visualizations}{%
\subsection{Need for informative visualizations}\label{need-for-informative-visualizations}}

\hypertarget{need-for-better-statistical-reporting}{%
\subsection{Need for better statistical reporting}\label{need-for-better-statistical-reporting}}

But why would combining statistical analysis with data visualization be helpful?
We list few reasons below-

\begin{itemize}
\tightlist
\item
  A recent survey (Nuijten, Hartgerink, van Assen, Epskamp, \& Wicherts, \protect\hyperlink{ref-nuijtenPrevalenceStatisticalReporting2016}{2016}) revealed that
  one in eight papers in major psychology journals contained a grossly
  inconsistent \emph{p}-value that may have affected the statistical conclusion.
  \texttt{ggstatsplot} helps avoid such reporting errors: Since the plot and the
  statistical analysis are yoked together, the chances of making an error in
  reporting the results are minimized. One need not write the results manually
  or copy-paste them from a different statistics software program (like SPSS,
  SAS, and so on).
\end{itemize}

\hypertarget{ggstatsplot-at-a-glance}{%
\section{\texorpdfstring{\texttt{ggstatsplot} at a glance}{ggstatsplot at a glance}}\label{ggstatsplot-at-a-glance}}

\hypertarget{summary-of-types-of-plots-included}{%
\subsection{Summary of types of plots included}\label{summary-of-types-of-plots-included}}

It produces a limited kinds of ready-made plots for the supported analyses:

\begin{longtable}[]{@{}lll@{}}
\toprule
\begin{minipage}[b]{0.17\columnwidth}\raggedright
Function\strut
\end{minipage} & \begin{minipage}[b]{0.25\columnwidth}\raggedright
Plot\strut
\end{minipage} & \begin{minipage}[b]{0.49\columnwidth}\raggedright
Description\strut
\end{minipage}\tabularnewline
\midrule
\endhead
\begin{minipage}[t]{0.17\columnwidth}\raggedright
\texttt{ggbetweenstats}\strut
\end{minipage} & \begin{minipage}[t]{0.25\columnwidth}\raggedright
\textbf{violin plots}\strut
\end{minipage} & \begin{minipage}[t]{0.49\columnwidth}\raggedright
for comparisons \emph{between} groups/conditions\strut
\end{minipage}\tabularnewline
\begin{minipage}[t]{0.17\columnwidth}\raggedright
\texttt{ggwithinstats}\strut
\end{minipage} & \begin{minipage}[t]{0.25\columnwidth}\raggedright
\textbf{violin plots}\strut
\end{minipage} & \begin{minipage}[t]{0.49\columnwidth}\raggedright
for comparisons \emph{within} groups/conditions\strut
\end{minipage}\tabularnewline
\begin{minipage}[t]{0.17\columnwidth}\raggedright
\texttt{gghistostats}\strut
\end{minipage} & \begin{minipage}[t]{0.25\columnwidth}\raggedright
\textbf{histograms}\strut
\end{minipage} & \begin{minipage}[t]{0.49\columnwidth}\raggedright
for distribution about numeric variable\strut
\end{minipage}\tabularnewline
\begin{minipage}[t]{0.17\columnwidth}\raggedright
\texttt{ggdotplotstats}\strut
\end{minipage} & \begin{minipage}[t]{0.25\columnwidth}\raggedright
\textbf{dot plots/charts}\strut
\end{minipage} & \begin{minipage}[t]{0.49\columnwidth}\raggedright
for distribution about labeled numeric variable\strut
\end{minipage}\tabularnewline
\begin{minipage}[t]{0.17\columnwidth}\raggedright
\texttt{ggpiestats}\strut
\end{minipage} & \begin{minipage}[t]{0.25\columnwidth}\raggedright
\textbf{pie charts}\strut
\end{minipage} & \begin{minipage}[t]{0.49\columnwidth}\raggedright
for categorical data\strut
\end{minipage}\tabularnewline
\begin{minipage}[t]{0.17\columnwidth}\raggedright
\texttt{ggbarstats}\strut
\end{minipage} & \begin{minipage}[t]{0.25\columnwidth}\raggedright
\textbf{bar charts}\strut
\end{minipage} & \begin{minipage}[t]{0.49\columnwidth}\raggedright
for categorical data\strut
\end{minipage}\tabularnewline
\begin{minipage}[t]{0.17\columnwidth}\raggedright
\texttt{ggscatterstats}\strut
\end{minipage} & \begin{minipage}[t]{0.25\columnwidth}\raggedright
\textbf{scatterplots}\strut
\end{minipage} & \begin{minipage}[t]{0.49\columnwidth}\raggedright
for correlations between two variables\strut
\end{minipage}\tabularnewline
\begin{minipage}[t]{0.17\columnwidth}\raggedright
\texttt{ggcorrmat}\strut
\end{minipage} & \begin{minipage}[t]{0.25\columnwidth}\raggedright
\textbf{correlation matrices}\strut
\end{minipage} & \begin{minipage}[t]{0.49\columnwidth}\raggedright
for correlations between multiple variables\strut
\end{minipage}\tabularnewline
\begin{minipage}[t]{0.17\columnwidth}\raggedright
\texttt{ggcoefstats}\strut
\end{minipage} & \begin{minipage}[t]{0.25\columnwidth}\raggedright
\textbf{dot-and-whisker plots}\strut
\end{minipage} & \begin{minipage}[t]{0.49\columnwidth}\raggedright
for regression models\strut
\end{minipage}\tabularnewline
\bottomrule
\end{longtable}

In addition to these basic plots, \texttt{ggstatsplot} also provides \textbf{\texttt{grouped\_}}
versions (see below) that makes it easy to repeat the same analysis for
any grouping variable.

\hypertarget{summary-of-types-of-statistical-analyses}{%
\subsection{Summary of types of statistical analyses}\label{summary-of-types-of-statistical-analyses}}

Most functions share a \texttt{type} (of test) argument that is helpful to specify the
type of statistical analysis:

\begin{itemize}
\tightlist
\item
  \texttt{"parametric"} (for \textbf{parametric} tests)
\item
  \texttt{"nonparametric"} (for \textbf{non-parametric} tests)
\item
  \texttt{"robust"} (for \textbf{robust} tests)
\item
  \texttt{"bayes"} (for \textbf{Bayes Factor} tests)
\end{itemize}

The table below summarizes all the different types of analyses
currently supported in this package-

\begin{longtable}[]{@{}llllll@{}}
\toprule
\begin{minipage}[b]{0.14\columnwidth}\raggedright
Functions\strut
\end{minipage} & \begin{minipage}[b]{0.35\columnwidth}\raggedright
Description\strut
\end{minipage} & \begin{minipage}[b]{0.08\columnwidth}\raggedright
Parametric\strut
\end{minipage} & \begin{minipage}[b]{0.10\columnwidth}\raggedright
Non-parametric\strut
\end{minipage} & \begin{minipage}[b]{0.08\columnwidth}\raggedright
Robust\strut
\end{minipage} & \begin{minipage}[b]{0.10\columnwidth}\raggedright
Bayes Factor\strut
\end{minipage}\tabularnewline
\midrule
\endhead
\begin{minipage}[t]{0.14\columnwidth}\raggedright
\texttt{ggbetweenstats}\strut
\end{minipage} & \begin{minipage}[t]{0.35\columnwidth}\raggedright
Between group/condition comparisons\strut
\end{minipage} & \begin{minipage}[t]{0.08\columnwidth}\raggedright
\textcolor{ForestGreen}{Yes}\strut
\end{minipage} & \begin{minipage}[t]{0.10\columnwidth}\raggedright
\textcolor{ForestGreen}{Yes}\strut
\end{minipage} & \begin{minipage}[t]{0.08\columnwidth}\raggedright
\textcolor{ForestGreen}{Yes}\strut
\end{minipage} & \begin{minipage}[t]{0.10\columnwidth}\raggedright
\textcolor{ForestGreen}{Yes}\strut
\end{minipage}\tabularnewline
\begin{minipage}[t]{0.14\columnwidth}\raggedright
\texttt{ggwithinstats}\strut
\end{minipage} & \begin{minipage}[t]{0.35\columnwidth}\raggedright
Within group/condition comparisons\strut
\end{minipage} & \begin{minipage}[t]{0.08\columnwidth}\raggedright
\textcolor{ForestGreen}{Yes}\strut
\end{minipage} & \begin{minipage}[t]{0.10\columnwidth}\raggedright
\textcolor{ForestGreen}{Yes}\strut
\end{minipage} & \begin{minipage}[t]{0.08\columnwidth}\raggedright
\textcolor{ForestGreen}{Yes}\strut
\end{minipage} & \begin{minipage}[t]{0.10\columnwidth}\raggedright
\textcolor{ForestGreen}{Yes}\strut
\end{minipage}\tabularnewline
\begin{minipage}[t]{0.14\columnwidth}\raggedright
\texttt{gghistostats}, \texttt{ggdotplotstats}\strut
\end{minipage} & \begin{minipage}[t]{0.35\columnwidth}\raggedright
Distribution of a numeric variable\strut
\end{minipage} & \begin{minipage}[t]{0.08\columnwidth}\raggedright
\textcolor{ForestGreen}{Yes}\strut
\end{minipage} & \begin{minipage}[t]{0.10\columnwidth}\raggedright
\textcolor{ForestGreen}{Yes}\strut
\end{minipage} & \begin{minipage}[t]{0.08\columnwidth}\raggedright
\textcolor{ForestGreen}{Yes}\strut
\end{minipage} & \begin{minipage}[t]{0.10\columnwidth}\raggedright
\textcolor{ForestGreen}{Yes}\strut
\end{minipage}\tabularnewline
\begin{minipage}[t]{0.14\columnwidth}\raggedright
\texttt{ggcorrmat}\strut
\end{minipage} & \begin{minipage}[t]{0.35\columnwidth}\raggedright
Correlation matrix\strut
\end{minipage} & \begin{minipage}[t]{0.08\columnwidth}\raggedright
\textcolor{ForestGreen}{Yes}\strut
\end{minipage} & \begin{minipage}[t]{0.10\columnwidth}\raggedright
\textcolor{ForestGreen}{Yes}\strut
\end{minipage} & \begin{minipage}[t]{0.08\columnwidth}\raggedright
\textcolor{ForestGreen}{Yes}\strut
\end{minipage} & \begin{minipage}[t]{0.10\columnwidth}\raggedright
\textcolor{red}{No}\strut
\end{minipage}\tabularnewline
\begin{minipage}[t]{0.14\columnwidth}\raggedright
\texttt{ggscatterstats}\strut
\end{minipage} & \begin{minipage}[t]{0.35\columnwidth}\raggedright
Correlation between two variables\strut
\end{minipage} & \begin{minipage}[t]{0.08\columnwidth}\raggedright
\textcolor{ForestGreen}{Yes}\strut
\end{minipage} & \begin{minipage}[t]{0.10\columnwidth}\raggedright
\textcolor{ForestGreen}{Yes}\strut
\end{minipage} & \begin{minipage}[t]{0.08\columnwidth}\raggedright
\textcolor{ForestGreen}{Yes}\strut
\end{minipage} & \begin{minipage}[t]{0.10\columnwidth}\raggedright
\textcolor{ForestGreen}{Yes}\strut
\end{minipage}\tabularnewline
\begin{minipage}[t]{0.14\columnwidth}\raggedright
\texttt{ggpiestats}, \texttt{ggbarstats}\strut
\end{minipage} & \begin{minipage}[t]{0.35\columnwidth}\raggedright
Association between categorical variables\strut
\end{minipage} & \begin{minipage}[t]{0.08\columnwidth}\raggedright
\textcolor{ForestGreen}{Yes}\strut
\end{minipage} & \begin{minipage}[t]{0.10\columnwidth}\raggedright
\texttt{NA}\strut
\end{minipage} & \begin{minipage}[t]{0.08\columnwidth}\raggedright
\texttt{NA}\strut
\end{minipage} & \begin{minipage}[t]{0.10\columnwidth}\raggedright
\textcolor{ForestGreen}{Yes}\strut
\end{minipage}\tabularnewline
\begin{minipage}[t]{0.14\columnwidth}\raggedright
\texttt{ggpiestats}, \texttt{ggbarstats}\strut
\end{minipage} & \begin{minipage}[t]{0.35\columnwidth}\raggedright
Equal proportions for categorical variable levels\strut
\end{minipage} & \begin{minipage}[t]{0.08\columnwidth}\raggedright
\textcolor{ForestGreen}{Yes}\strut
\end{minipage} & \begin{minipage}[t]{0.10\columnwidth}\raggedright
\texttt{NA}\strut
\end{minipage} & \begin{minipage}[t]{0.08\columnwidth}\raggedright
\texttt{NA}\strut
\end{minipage} & \begin{minipage}[t]{0.10\columnwidth}\raggedright
\textcolor{ForestGreen}{Yes}\strut
\end{minipage}\tabularnewline
\begin{minipage}[t]{0.14\columnwidth}\raggedright
\texttt{ggcoefstats}\strut
\end{minipage} & \begin{minipage}[t]{0.35\columnwidth}\raggedright
Regression model coefficients\strut
\end{minipage} & \begin{minipage}[t]{0.08\columnwidth}\raggedright
\textcolor{ForestGreen}{Yes}\strut
\end{minipage} & \begin{minipage}[t]{0.10\columnwidth}\raggedright
\textcolor{red}{No}\strut
\end{minipage} & \begin{minipage}[t]{0.08\columnwidth}\raggedright
\textcolor{ForestGreen}{Yes}\strut
\end{minipage} & \begin{minipage}[t]{0.10\columnwidth}\raggedright
\textcolor{red}{No}\strut
\end{minipage}\tabularnewline
\bottomrule
\end{longtable}

In the following sections, we will discuss at depth justification for why the
plots have been designed in certain ways and what principles were followed to
report statistical details on the plots.

\hypertarget{graphic-design-principles}{%
\section{Graphic design principles}\label{graphic-design-principles}}

\hypertarget{graphical-perception}{%
\subsection{Graphical perception}\label{graphical-perception}}

Graphical perception involves visual decoding of the encoded information in
graphs. \texttt{ggstatsplot} incorporates the paradigm proposed in
((Cleveland, \protect\hyperlink{ref-clevelandElementsGraphingData1985}{1985}), Chapter 4) to facilitate making visual
judgments about quantitative information effortless and almost instantaneous.
Based on experiments, Cleveland proposes that there are ten elementary
graphical-perception tasks that we perform to visually decode quantitative
information in graphs (organized from most to least accurate;
(Cleveland, \protect\hyperlink{ref-clevelandElementsGraphingData1985}{1985}), p.254)-

\begin{itemize}
\tightlist
\item
  Position along a common scale
\item
  Position along identical, non-aligned scales
\item
  Length
\item
  Angle (Slope)
\item
  Area
\item
  Volume
\item
  Color hue
\end{itemize}

So the key principle of Cleveland's paradigm for data display is-

\begin{quote}
``We should encode data on a graph so that the visual decoding involves
{[}graphical-perception{]} tasks as high in the ordering as possible.''
\end{quote}

For example, decoding the data point values in \texttt{ggbetweenstats} requires
position judgments along a common scale (Figure \ref{fig:fig1}):

\begin{Shaded}
\begin{Highlighting}[]
\CommentTok{# for reproducibility}
\KeywordTok{set.seed}\NormalTok{(}\DecValTok{123}\NormalTok{)}

\CommentTok{# plot}
\NormalTok{ggstatsplot}\OperatorTok{::}\KeywordTok{ggbetweenstats}\NormalTok{(}
  \DataTypeTok{data =}\NormalTok{ dplyr}\OperatorTok{::}\KeywordTok{filter}\NormalTok{(}
    \DataTypeTok{.data =}\NormalTok{ ggstatsplot}\OperatorTok{::}\NormalTok{movies_long,}
\NormalTok{    genre }\OperatorTok\StringTok{ }\KeywordTok{c}\NormalTok{(}\StringTok{"Action"}\NormalTok{, }\StringTok{"Action Comedy"}\NormalTok{, }\StringTok{"Action Drama"}\NormalTok{, }\StringTok{"Comedy"}\NormalTok{)}
\NormalTok{  ),}
  \DataTypeTok{x =}\NormalTok{ genre,}
  \DataTypeTok{y =}\NormalTok{ rating,}
  \DataTypeTok{title =} \StringTok{"IMDB rating by film genre"}\NormalTok{,}
  \DataTypeTok{xlab =} \StringTok{"Genre"}\NormalTok{,}
  \DataTypeTok{ylab =} \StringTok{"IMDB rating (average)"}\NormalTok{,}
  \DataTypeTok{pairwise.comparisons =} \OtherTok{TRUE}\NormalTok{,}
  \DataTypeTok{p.adjust.method =} \StringTok{"bonferroni"}\NormalTok{,}
  \DataTypeTok{ggtheme =}\NormalTok{ hrbrthemes}\OperatorTok{::}\KeywordTok{theme_ipsum_tw}\NormalTok{(),}
  \DataTypeTok{ggstatsplot.layer =} \OtherTok{FALSE}\NormalTok{,}
  \DataTypeTok{outlier.tagging =} \OtherTok{TRUE}\NormalTok{,}
  \DataTypeTok{outlier.label =}\NormalTok{ title,}
  \DataTypeTok{messages =} \OtherTok{FALSE}
\NormalTok{)}
\end{Highlighting}
\end{Shaded}

\begin{figure}[H]
\includegraphics[width=1\linewidth]{./figures/paper-fig1-1} \caption{Note that assessing differences in mean values between groups has been made easier with the help of \textit{position} of data points along a common scale (the Y-axis) and labels.}\label{fig:fig1}
\end{figure}

There are few instances where \texttt{ggstatsplot} diverges from recommendations made
in Cleveland's paradigm:

\begin{itemize}
\tightlist
\item
  For the categorical/nominal data, \texttt{ggstatsplot} uses pie charts (see Figure
  \ref{fig:fig2}) which rely on \emph{angle} judgments, which are less accurate (as
  compared to bar graphs, e.g., which require \emph{position} judgments). This
  shortcoming is assuaged to some degree by using plenty of labels that
  describe percentages for all slices. This makes angle judgment unnecessary
  and pre-vacates any concerns about inaccurate judgments about percentages.
  Additionally, it also provides alternative function to \texttt{ggpiestats} for
  working with categorical variables: \texttt{ggbarstats}.
\end{itemize}

\begin{Shaded}
\begin{Highlighting}[]
\CommentTok{# for reproducibility}
\KeywordTok{set.seed}\NormalTok{(}\DecValTok{123}\NormalTok{)}

\CommentTok{# plot}
\NormalTok{ggstatsplot}\OperatorTok{::}\KeywordTok{ggpiestats}\NormalTok{(}
  \DataTypeTok{data =}\NormalTok{ ggstatsplot}\OperatorTok{::}\NormalTok{movies_long,}
  \DataTypeTok{x =}\NormalTok{ genre,}
  \DataTypeTok{y =}\NormalTok{ mpaa,}
  \DataTypeTok{title =} \StringTok{"Distribution of MPAA ratings by film genre"}\NormalTok{,}
  \DataTypeTok{legend.title =} \StringTok{"layout"}\NormalTok{,}
  \DataTypeTok{caption =} \KeywordTok{substitute}\NormalTok{(}\KeywordTok{paste}\NormalTok{(}
    \KeywordTok{italic}\NormalTok{(}\StringTok{"MPAA"}\NormalTok{), }\StringTok{": Motion Picture Association of America"}
\NormalTok{  )),}
  \DataTypeTok{palette =} \StringTok{"Paired"}\NormalTok{,}
  \DataTypeTok{messages =} \OtherTok{FALSE}
\NormalTok{)}
\end{Highlighting}
\end{Shaded}

\begin{figure}[H]
\includegraphics[width=1\linewidth]{./figures/paper-fig2-1} \caption{Pie charts don't follow Cleveland's paradigm to data display because they rely on less accurate angle judgments. `ggstatsplot` sidesteps this issue by always labelling percentages for pie slices, which makes angle judgments unnecessary.}\label{fig:fig2}
\end{figure}

\begin{itemize}
\tightlist
\item
  Cleveland's paradigm also emphasizes that \emph{superposition} of data is better
  than \emph{juxtaposition} ((Cleveland, \protect\hyperlink{ref-clevelandElementsGraphingData1985}{1985}), p.201) because
  this allows for a more incisive comparison of the values from different
  parts of the dataset. This recommendation is violated in all \texttt{grouped\_}
  variants of the function (see Figure \ref{fig:fig3}). Note that the range
  for Y-axes are no longer the same across juxtaposed subplots and so visually
  comparing the data becomes difficult. On the other hand, in the superposed
  plot, all data have the same range and coloring different parts makes the
  visual discrimination of different components of the data, and their
  comparison, easier. But the goal of \texttt{grouped\_} variants of functions is to
  not only show different aspects of the data but also to run statistical
  tests and showing detailed results for all aspects of the data in a
  superposed plot is difficult. Therefore, this is a compromise \texttt{ggstatsplot}
  is comfortable with, at least to produce plots for quick exploration of
  different aspects of the data.
\end{itemize}

\begin{Shaded}
\begin{Highlighting}[]
\CommentTok{# for reproducibility}
\KeywordTok{set.seed}\NormalTok{(}\DecValTok{123}\NormalTok{)}
\KeywordTok{library}\NormalTok{(ggplot2)}

\CommentTok{# creating a smaller dataframe}
\NormalTok{df <-}\StringTok{ }\NormalTok{dplyr}\OperatorTok{::}\KeywordTok{filter}\NormalTok{(ggstatsplot}\OperatorTok{::}\NormalTok{movies_long, genre }\OperatorTok\StringTok{ }\KeywordTok{c}\NormalTok{(}\StringTok{"Comedy"}\NormalTok{, }\StringTok{"Drama"}\NormalTok{))}

\CommentTok{# plot}
\NormalTok{ggstatsplot}\OperatorTok{::}\KeywordTok{combine_plots}\NormalTok{(}
  \CommentTok{# plot 1: superposition}
  \KeywordTok{ggplot}\NormalTok{(}\DataTypeTok{data =}\NormalTok{ df, }\DataTypeTok{mapping =}\NormalTok{ ggplot2}\OperatorTok{::}\KeywordTok{aes}\NormalTok{(}\DataTypeTok{x =}\NormalTok{ length, }\DataTypeTok{y =}\NormalTok{ rating, }\DataTypeTok{color =}\NormalTok{ genre)) }\OperatorTok{+}
\StringTok{    }\KeywordTok{geom_jitter}\NormalTok{(}\DataTypeTok{size =} \DecValTok{3}\NormalTok{, }\DataTypeTok{alpha =} \FloatTok{0.5}\NormalTok{) }\OperatorTok{+}
\StringTok{    }\KeywordTok{geom_smooth}\NormalTok{(}\DataTypeTok{method =} \StringTok{"lm"}\NormalTok{) }\OperatorTok{+}
\StringTok{    }\KeywordTok{labs}\NormalTok{(}\DataTypeTok{title =} \StringTok{"superposition (recommended in Cleveland's paradigm)"}\NormalTok{) }\OperatorTok{+}
\StringTok{    }\NormalTok{ggstatsplot}\OperatorTok{::}\KeywordTok{theme_ggstatsplot}\NormalTok{(),}
  \CommentTok{# plot 2: juxtaposition}
\NormalTok{  ggstatsplot}\OperatorTok{::}\KeywordTok{grouped_ggscatterstats}\NormalTok{(}
    \DataTypeTok{data =}\NormalTok{ df,}
    \DataTypeTok{x =}\NormalTok{ length,}
    \DataTypeTok{y =}\NormalTok{ rating,}
    \DataTypeTok{grouping.var =}\NormalTok{ genre,}
    \DataTypeTok{marginal =} \OtherTok{FALSE}\NormalTok{,}
    \DataTypeTok{messages =} \OtherTok{FALSE}\NormalTok{,}
    \DataTypeTok{title.prefix =} \StringTok{"Genre"}\NormalTok{,}
    \DataTypeTok{title.text =} \StringTok{"juxtaposition (`ggstatsplot` implementation in `grouped_` functions)"}\NormalTok{,}
    \DataTypeTok{title.size =} \DecValTok{12}
\NormalTok{  ),}
  \CommentTok{# combine for comparison}
  \DataTypeTok{title.text =} \StringTok{"Two ways to compare different aspects of data"}\NormalTok{,}
  \DataTypeTok{nrow =} \DecValTok{2}\NormalTok{,}
  \DataTypeTok{labels =} \KeywordTok{c}\NormalTok{(}\StringTok{"(a)"}\NormalTok{, }\StringTok{"(b)"}\NormalTok{)}
\NormalTok{)}
\end{Highlighting}
\end{Shaded}

\begin{figure}[H]
\includegraphics[width=1\linewidth]{./figures/paper-fig3-1} \caption{Comparing different aspects of data is much more accurate in (\textit{a}) a \textit{superposed} plot, which is recommended in Cleveland's paradigm, than in (\textit{b}) a \textit{juxtaposed} plot, which is how it is implemented in `ggstatsplot` package. This is because displaying detailed results from statistical tests would be difficult in a superposed plot.}\label{fig:fig3}
\end{figure}

The \texttt{grouped\_} plots follow the \emph{Shrink Principle}
((Tufte, \protect\hyperlink{ref-tufteVisualDisplayQuantitative2001}{2001}), p.166-7) for high-information graphics,
which dictates that the data density and the size of the data matrix can be
maximized to exploit maximum resolution of the available data-display
technology. Given the large maximum resolution afforded by most computer
monitors today, saving \texttt{grouped\_} plots with appropriate resolution ensures no
loss in legibility with reduced graphics area.

\hypertarget{graphical-excellence}{%
\subsection{Graphical excellence}\label{graphical-excellence}}

Graphical excellence consists of communicating complex ideas with clarity and in
a way that the viewer understands the greatest number of ideas in a short amount
of time all the while not quoting the data out of context. The package follows
the principles for \emph{graphical integrity} (Tufte, \protect\hyperlink{ref-tufteVisualDisplayQuantitative2001}{2001}):

\begin{itemize}
\item
  The physical representation of numbers is proportional to the numerical
  quantities they represent (e.g., Figure \ref{fig:fig1} and Figure
  \ref{fig:fig2} show how means (in \texttt{ggbetweenstats}) or percentages
  (\texttt{ggpiestats}) are proportional to the vertical distance or the area,
  respectively).
\item
  All important events in the data have clear, detailed, and thorough labeling
  (e.g., Figure \ref{fig:fig1} plot shows how \texttt{ggbetweenstats} labels means,
  sample size information, outliers, and pairwise comparisons; same can be
  appreciated for \texttt{ggpiestats} in Figure \ref{fig:fig2} and \texttt{gghistostats} in
  Figure \ref{fig:fig5}). Note that data labels in the data region are
  designed in a way that they don't interfere with our ability to assess the
  overall pattern of the data ((Cleveland, \protect\hyperlink{ref-clevelandElementsGraphingData1985}{1985});
  p.44-45). This is achieved by using \texttt{ggrepel} package to place labels in a
  way that reduces their visual prominence.
\item
  None of the plots have \emph{design} variation (e.g., abrupt change in scales)
  over the surface of a same graphic because this can lead to a false
  impression about variation in \emph{data}.
\item
  The number of information-carrying dimensions never exceed the number of
  dimensions in the data (e.g., using area to show one-dimensional data).
\item
  All plots are designed to have no \textbf{chartjunk} (like moiré vibrations, fake
  perspective, dark grid lines, etc.)
  ((Tufte, \protect\hyperlink{ref-tufteVisualDisplayQuantitative2001}{2001}), Chapter 5).
\end{itemize}

There are some instances where \texttt{ggstatsplot} graphs don't follow principles of
clean graphics, as formulated in the Tufte theory of data graphics
((Tufte, \protect\hyperlink{ref-tufteVisualDisplayQuantitative2001}{2001}), Chapter 4). The theory has four key
principles:

\begin{enumerate}
\def\labelenumi{\arabic{enumi}.}
\tightlist
\item
  Above all else show the data.
\item
  Maximize the data-ink ratio.
\item
  Erase non-data-ink.
\item
  Erase redundant data-ink, within reason.
\end{enumerate}

In particular, default plots in \texttt{ggstatsplot} can sometimes violate one of the
principles from 2-4. According to these principles, every bit of ink should have
reason for its inclusion in the graphic and should convey some new information
to the viewer. If not, such ink should be removed. One instance of this is
bilateral symmetry of data measures. For example, in Figure \ref{fig:fig1}, we
can see that both the box and violin plots are mirrored, which consumes twice
the space in the graphic without adding any new information. But this redundancy
is tolerated for the sake of beauty that such symmetrical shapes can bring to
the graphic. Even Tufte admits that efficiency is but one consideration in the
design of statistical graphics ((Tufte, \protect\hyperlink{ref-tufteVisualDisplayQuantitative2001}{2001}),
p.~137). Additionally, these principles were formulated in an era in which
computer graphics had yet to revolutionize the ease with which graphics could be
produced and thus some of the concerns about minimizing data-ink for easier
production of graphics are not as relevant as they were.

\hypertarget{statistical-variation}{%
\subsection{Statistical variation}\label{statistical-variation}}

One of the important functions of a plot is to show the variation in the data,
which comes in two forms:

\begin{itemize}
\tightlist
\item
  \textbf{Measurement noise}: In \texttt{ggstatsplot}, the actual variation in
  measurements is shown by plotting a combination of (jittered) raw data
  points with a boxplot laid on top (Figure \ref{fig:fig1}) or a histogram
  (Figure \ref{fig:fig5}). None of the plots, where empirical distribution of
  the data is concerned, show the sample standard deviation because they are
  poor at conveying information about limits of the sample and presence of
  outliers ((Cleveland, \protect\hyperlink{ref-clevelandElementsGraphingData1985}{1985}), p.220).
\end{itemize}

\begin{Shaded}
\begin{Highlighting}[]
\CommentTok{# for reproducibility}
\KeywordTok{set.seed}\NormalTok{(}\DecValTok{123}\NormalTok{)}

\CommentTok{# plot}
\NormalTok{ggstatsplot}\OperatorTok{::}\KeywordTok{gghistostats}\NormalTok{(}
  \DataTypeTok{data =}\NormalTok{ morley,}
  \DataTypeTok{x =}\NormalTok{ Speed,}
  \DataTypeTok{test.value =} \DecValTok{792}\NormalTok{,}
  \DataTypeTok{test.value.line =} \OtherTok{TRUE}\NormalTok{,}
  \DataTypeTok{xlab =} \StringTok{"Speed of light (km/sec, with 299000 subtracted)"}\NormalTok{,}
  \DataTypeTok{title =} \StringTok{"Distribution of measured Speed of light"}\NormalTok{,}
  \DataTypeTok{caption =} \StringTok{"Note: Data collected across 5 experiments (20 measurements each)"}\NormalTok{,}
  \DataTypeTok{messages =} \OtherTok{FALSE}
\NormalTok{)}
\end{Highlighting}
\end{Shaded}

\begin{figure}[H]
\includegraphics[width=1\linewidth]{./figures/paper-fig5-1} \caption{Distribution of a variable shown using `gghistostats`.}\label{fig:fig5}
\end{figure}

\begin{itemize}
\tightlist
\item
  \textbf{Sample-to-sample statistic variation}: Although, traditionally, this
  variation has been shown using the standard error of the mean (SEM) of the
  statistic, \texttt{ggstatsplot} plots instead use 95\% confidence intervals (e.g.,
  Figure \ref{fig:fig6}). This is because the interval formed by error bars
  correspond to a 68\% confidence interval, which is not a particularly
  interesting interval ((Cleveland, \protect\hyperlink{ref-clevelandElementsGraphingData1985}{1985}), p.222-225).
\end{itemize}

\begin{Shaded}
\begin{Highlighting}[]
\CommentTok{# for reproducibility}
\KeywordTok{set.seed}\NormalTok{(}\DecValTok{123}\NormalTok{)}

\CommentTok{# creating model object}
\NormalTok{mod <-}\StringTok{ }\NormalTok{lme4}\OperatorTok{::}\KeywordTok{lmer}\NormalTok{(}
  \DataTypeTok{formula =}\NormalTok{ total.fruits }\OperatorTok{~}\StringTok{ }\NormalTok{nutrient }\OperatorTok{+}\StringTok{ }\NormalTok{rack }\OperatorTok{+}\StringTok{ }\NormalTok{(nutrient }\OperatorTok{|}\StringTok{ }\NormalTok{gen),}
  \DataTypeTok{data =}\NormalTok{ lme4}\OperatorTok{::}\NormalTok{Arabidopsis}
\NormalTok{)}

\CommentTok{# plot}
\NormalTok{ggstatsplot}\OperatorTok{::}\KeywordTok{ggcoefstats}\NormalTok{(}\DataTypeTok{x =}\NormalTok{ mod)}
\end{Highlighting}
\end{Shaded}

\begin{figure}[H]
\includegraphics[width=1\linewidth]{./figures/paper-fig6-1} \caption{Sample-to-sample variation in regression estimates is displayed using confidence intervals in `ggcoefstats`.}\label{fig:fig6}
\end{figure}

\hypertarget{statistical-analysis}{%
\section{Statistical analysis}\label{statistical-analysis}}

\hypertarget{data-requirements}{%
\subsection{Data requirements}\label{data-requirements}}

As an extension of \texttt{ggplot2}, \texttt{ggstatsplot} has the same expectations about the
structure of the data. More specifically,

\begin{itemize}
\item
  The data should be organized following the principles of \emph{tidy data}, which
  specify how statistical structure of a data frame (variables and
  observations) should be mapped to physical structure (columns and rows).
  More specifically, tidy data means all variables have their own columns and
  each row corresponds to a unique observation ((Wickham, \protect\hyperlink{ref-wickhamTidyData2014}{2014})).
\item
  All \texttt{ggstatsplot} functions remove \texttt{NA}s from variables of interest (similar
  to \texttt{ggplot2}; (Wickham, \protect\hyperlink{ref-wickhamGgplot2ElegantGraphics2016}{2016}), p.207) in the data and
  display total sample size (\emph{n}, either observations for between-subjects or
  pairs for within-subjects designs) in the subtitle to inform the user/reader
  about the number of observations included for both the statistical analysis
  and the visualization. But, when sample sizes differ \emph{across} tests in the
  same function, \texttt{ggstatsplot} makes an effort to inform the user of this
  aspect. For example, \texttt{ggcorrmat} features several correlation test pairs
  and, depending on variables in a given pair, the sample sizes may vary
  (Figure \ref{fig:fig4}).
\end{itemize}

\begin{Shaded}
\begin{Highlighting}[]
\CommentTok{# for reproducibility}
\KeywordTok{set.seed}\NormalTok{(}\DecValTok{123}\NormalTok{)}

\CommentTok{# creating a new dataset without any NAs in variables of interest}
\NormalTok{msleep_no_na <-}
\StringTok{  }\NormalTok{dplyr}\OperatorTok{::}\KeywordTok{filter}\NormalTok{(}
    \DataTypeTok{.data =}\NormalTok{ ggplot2}\OperatorTok{::}\NormalTok{msleep,}
    \OperatorTok{!}\KeywordTok{is.na}\NormalTok{(sleep_rem), }\OperatorTok{!}\KeywordTok{is.na}\NormalTok{(awake), }\OperatorTok{!}\KeywordTok{is.na}\NormalTok{(brainwt), }\OperatorTok{!}\KeywordTok{is.na}\NormalTok{(bodywt)}
\NormalTok{  )}

\CommentTok{# variable names vector}
\NormalTok{var_names <-}\StringTok{ }\KeywordTok{c}\NormalTok{(}\StringTok{"REM sleep"}\NormalTok{, }\StringTok{"time awake"}\NormalTok{, }\StringTok{"brain weight"}\NormalTok{, }\StringTok{"body weight"}\NormalTok{)}

\CommentTok{# combining two plots using helper function in `ggstatsplot`}
\NormalTok{ggstatsplot}\OperatorTok{::}\KeywordTok{combine_plots}\NormalTok{(}
  \DataTypeTok{plotlist =}\NormalTok{ purrr}\OperatorTok{::}\KeywordTok{pmap}\NormalTok{(}
    \DataTypeTok{.l =} \KeywordTok{list}\NormalTok{(}\DataTypeTok{data =} \KeywordTok{list}\NormalTok{(msleep_no_na, ggplot2}\OperatorTok{::}\NormalTok{msleep)),}
    \DataTypeTok{.f =}\NormalTok{ ggstatsplot}\OperatorTok{::}\NormalTok{ggcorrmat,}
    \DataTypeTok{p.adjust.method =} \StringTok{"holm"}\NormalTok{,}
    \DataTypeTok{cor.vars =} \KeywordTok{c}\NormalTok{(sleep_rem, awake}\OperatorTok{:}\NormalTok{bodywt),}
    \DataTypeTok{cor.vars.names =}\NormalTok{ var_names,}
    \DataTypeTok{matrix.type =} \StringTok{"upper"}\NormalTok{,}
    \DataTypeTok{colors =} \KeywordTok{c}\NormalTok{(}\StringTok{"#B2182B"}\NormalTok{, }\StringTok{"white"}\NormalTok{, }\StringTok{"#4D4D4D"}\NormalTok{),}
    \DataTypeTok{title =} \StringTok{"Correlalogram for mammals sleep dataset"}\NormalTok{,}
    \DataTypeTok{subtitle =} \StringTok{"sleep units: hours; weight units: kilograms"}\NormalTok{,}
    \DataTypeTok{messages =} \OtherTok{FALSE}
\NormalTok{  ),}
  \DataTypeTok{labels =} \KeywordTok{c}\NormalTok{(}\StringTok{"(a)"}\NormalTok{, }\StringTok{"(b)"}\NormalTok{),}
  \DataTypeTok{nrow =} \DecValTok{1}
\NormalTok{)}
\end{Highlighting}
\end{Shaded}

\begin{figure}[H]
\includegraphics[width=1\linewidth]{./figures/paper-fig4-1} \caption{`ggstatsplot` functions remove `NA`s from variables of interest and display total sample size \textit{n}, but they can give more nuanced information about sample sizes when \textit{n} differs across tests. For example, `ggcorrmat` will display (\textit{a}) only one total sample size once when no `NA`s present, but (\textit{b}) will instead show minimum, median, and maximum sample sizes across all correlation tests when `NA`s are present across correlation variables.}\label{fig:fig4}
\end{figure}

\hypertarget{statistical-reporting}{%
\subsection{Statistical reporting}\label{statistical-reporting}}

The default setting in \texttt{ggstatsplot} is to produce plots with statistical
details included. Most often than not, these results are displayed as a \texttt{subtitle}
in the plot. Great care has been taken into which details are included in
statistical reporting and why. For example, the template below is used to show
results from Yuen's test for trimmed means (robust \emph{t}-test):

\begin{figure}
\centering
\includegraphics{figures/stats_reporting_format.png}
\caption{Template for reporting statistical details}
\end{figure}

APA guidelines (Association, \protect\hyperlink{ref-associationPublicationManualAmerican2009}{2009}) are followed by
default while reporting statistical details:

\begin{itemize}
\tightlist
\item
  Percentages are displayed with no decimal places (Figure \ref{fig:fig2}).
\item
  Correlations, \emph{t}-tests, and \(\chi^2\)-tests are reported with the degrees
  of freedom in parentheses and the significance level (Figure \ref{fig:fig4},
  Figure \ref{fig:fig3}, Figure \ref{fig:fig5}).
\item
  ANOVAs are reported with two degrees of freedom and the significance level
  (Figure \ref{fig:fig1}).
\item
  Regression results are presented with the unstandardized or standardized
  estimate (beta), whichever was specified by the user, along with the
  statistic (depending on the model, this can be a \emph{t}, \emph{F}, or \emph{z} statistic)
  and the corresponding significance level (Figure \ref{fig:fig6}).
\item
  With the exception of \emph{p}-values, most statistics are rounded to two decimal
  places by default.
\end{itemize}

\hypertarget{statistical-tests}{%
\subsection{Statistical tests:}\label{statistical-tests}}

Here is a summary table of all the statistical tests currently supported across
various functions:

\begin{longtable}[]{@{}lllll@{}}
\toprule
\begin{minipage}[b]{0.20\columnwidth}\raggedright
Functions\strut
\end{minipage} & \begin{minipage}[b]{0.16\columnwidth}\raggedright
Type\strut
\end{minipage} & \begin{minipage}[b]{0.31\columnwidth}\raggedright
Test\strut
\end{minipage} & \begin{minipage}[b]{0.12\columnwidth}\raggedright
Effect size\strut
\end{minipage} & \begin{minipage}[b]{0.07\columnwidth}\raggedright
95\% CI available?\strut
\end{minipage}\tabularnewline
\midrule
\endhead
\begin{minipage}[t]{0.20\columnwidth}\raggedright
\texttt{ggbetweenstats} (2 groups)\strut
\end{minipage} & \begin{minipage}[t]{0.16\columnwidth}\raggedright
Parametric\strut
\end{minipage} & \begin{minipage}[t]{0.31\columnwidth}\raggedright
Student's and Welch's \emph{t}-test\strut
\end{minipage} & \begin{minipage}[t]{0.12\columnwidth}\raggedright
Cohen's \emph{d}, Hedge's \emph{g}\strut
\end{minipage} & \begin{minipage}[t]{0.07\columnwidth}\raggedright
\textcolor{ForestGreen}{$\checkmark$}\strut
\end{minipage}\tabularnewline
\begin{minipage}[t]{0.20\columnwidth}\raggedright
\texttt{ggbetweenstats} (\textgreater{} 2 groups)\strut
\end{minipage} & \begin{minipage}[t]{0.16\columnwidth}\raggedright
Parametric\strut
\end{minipage} & \begin{minipage}[t]{0.31\columnwidth}\raggedright
Fisher's and Welch's one-way ANOVA\strut
\end{minipage} & \begin{minipage}[t]{0.12\columnwidth}\raggedright
\[\eta^2, \eta^2_p, \omega^2, \omega^2_p\]\strut
\end{minipage} & \begin{minipage}[t]{0.07\columnwidth}\raggedright
\textcolor{ForestGreen}{$\checkmark$}\strut
\end{minipage}\tabularnewline
\begin{minipage}[t]{0.20\columnwidth}\raggedright
\texttt{ggbetweenstats} (2 groups)\strut
\end{minipage} & \begin{minipage}[t]{0.16\columnwidth}\raggedright
Non-parametric\strut
\end{minipage} & \begin{minipage}[t]{0.31\columnwidth}\raggedright
Mann-Whitney \emph{U}-test\strut
\end{minipage} & \begin{minipage}[t]{0.12\columnwidth}\raggedright
\emph{r}\strut
\end{minipage} & \begin{minipage}[t]{0.07\columnwidth}\raggedright
\textcolor{ForestGreen}{$\checkmark$}\strut
\end{minipage}\tabularnewline
\begin{minipage}[t]{0.20\columnwidth}\raggedright
\texttt{ggbetweenstats} (\textgreater{} 2 groups)\strut
\end{minipage} & \begin{minipage}[t]{0.16\columnwidth}\raggedright
Non-parametric\strut
\end{minipage} & \begin{minipage}[t]{0.31\columnwidth}\raggedright
Kruskal-Wallis Rank Sum Test\strut
\end{minipage} & \begin{minipage}[t]{0.12\columnwidth}\raggedright
\[\epsilon^2\]\strut
\end{minipage} & \begin{minipage}[t]{0.07\columnwidth}\raggedright
\textcolor{ForestGreen}{$\checkmark$}\strut
\end{minipage}\tabularnewline
\begin{minipage}[t]{0.20\columnwidth}\raggedright
\texttt{ggbetweenstats} (2 groups)\strut
\end{minipage} & \begin{minipage}[t]{0.16\columnwidth}\raggedright
Robust\strut
\end{minipage} & \begin{minipage}[t]{0.31\columnwidth}\raggedright
Yuen's test for trimmed means\strut
\end{minipage} & \begin{minipage}[t]{0.12\columnwidth}\raggedright
\[\xi\]\strut
\end{minipage} & \begin{minipage}[t]{0.07\columnwidth}\raggedright
\textcolor{ForestGreen}{$\checkmark$}\strut
\end{minipage}\tabularnewline
\begin{minipage}[t]{0.20\columnwidth}\raggedright
\texttt{ggbetweenstats} (\textgreater{} 2 groups)\strut
\end{minipage} & \begin{minipage}[t]{0.16\columnwidth}\raggedright
Robust\strut
\end{minipage} & \begin{minipage}[t]{0.31\columnwidth}\raggedright
Heteroscedastic one-way ANOVA for trimmed means\strut
\end{minipage} & \begin{minipage}[t]{0.12\columnwidth}\raggedright
\[\xi\]\strut
\end{minipage} & \begin{minipage}[t]{0.07\columnwidth}\raggedright
\textcolor{ForestGreen}{$\checkmark$}\strut
\end{minipage}\tabularnewline
\begin{minipage}[t]{0.20\columnwidth}\raggedright
\texttt{ggwithinstats} (2 groups)\strut
\end{minipage} & \begin{minipage}[t]{0.16\columnwidth}\raggedright
Parametric\strut
\end{minipage} & \begin{minipage}[t]{0.31\columnwidth}\raggedright
Student's \emph{t}-test\strut
\end{minipage} & \begin{minipage}[t]{0.12\columnwidth}\raggedright
Cohen's \emph{d}, Hedge's \emph{g}\strut
\end{minipage} & \begin{minipage}[t]{0.07\columnwidth}\raggedright
\textcolor{ForestGreen}{$\checkmark$}\strut
\end{minipage}\tabularnewline
\begin{minipage}[t]{0.20\columnwidth}\raggedright
\texttt{ggwithinstats} (\textgreater{} 2 groups)\strut
\end{minipage} & \begin{minipage}[t]{0.16\columnwidth}\raggedright
Parametric\strut
\end{minipage} & \begin{minipage}[t]{0.31\columnwidth}\raggedright
Fisher's one-way repeated measures ANOVA\strut
\end{minipage} & \begin{minipage}[t]{0.12\columnwidth}\raggedright
\[\eta^2_p, \omega^2\]\strut
\end{minipage} & \begin{minipage}[t]{0.07\columnwidth}\raggedright
\textcolor{ForestGreen}{$\checkmark$}\strut
\end{minipage}\tabularnewline
\begin{minipage}[t]{0.20\columnwidth}\raggedright
\texttt{ggwithinstats} (2 groups)\strut
\end{minipage} & \begin{minipage}[t]{0.16\columnwidth}\raggedright
Non-parametric\strut
\end{minipage} & \begin{minipage}[t]{0.31\columnwidth}\raggedright
Wilcoxon signed-rank test\strut
\end{minipage} & \begin{minipage}[t]{0.12\columnwidth}\raggedright
\emph{r}\strut
\end{minipage} & \begin{minipage}[t]{0.07\columnwidth}\raggedright
\textcolor{ForestGreen}{$\checkmark$}\strut
\end{minipage}\tabularnewline
\begin{minipage}[t]{0.20\columnwidth}\raggedright
\texttt{ggwithinstats} (\textgreater{} 2 groups)\strut
\end{minipage} & \begin{minipage}[t]{0.16\columnwidth}\raggedright
Non-parametric\strut
\end{minipage} & \begin{minipage}[t]{0.31\columnwidth}\raggedright
Friedman rank sum test\strut
\end{minipage} & \begin{minipage}[t]{0.12\columnwidth}\raggedright
\[W_{Kendall}\]\strut
\end{minipage} & \begin{minipage}[t]{0.07\columnwidth}\raggedright
\textcolor{ForestGreen}{$\checkmark$}\strut
\end{minipage}\tabularnewline
\begin{minipage}[t]{0.20\columnwidth}\raggedright
\texttt{ggwithinstats} (2 groups)\strut
\end{minipage} & \begin{minipage}[t]{0.16\columnwidth}\raggedright
Robust\strut
\end{minipage} & \begin{minipage}[t]{0.31\columnwidth}\raggedright
Yuen's test on trimmed means for dependent samples\strut
\end{minipage} & \begin{minipage}[t]{0.12\columnwidth}\raggedright
\[\xi\]\strut
\end{minipage} & \begin{minipage}[t]{0.07\columnwidth}\raggedright
\textcolor{ForestGreen}{$\checkmark$}\strut
\end{minipage}\tabularnewline
\begin{minipage}[t]{0.20\columnwidth}\raggedright
\texttt{ggwithinstats} (\textgreater{} 2 groups)\strut
\end{minipage} & \begin{minipage}[t]{0.16\columnwidth}\raggedright
Robust\strut
\end{minipage} & \begin{minipage}[t]{0.31\columnwidth}\raggedright
Heteroscedastic one-way repeated measures ANOVA for trimmed means\strut
\end{minipage} & \begin{minipage}[t]{0.12\columnwidth}\raggedright
\textcolor{red}{$\times$}\strut
\end{minipage} & \begin{minipage}[t]{0.07\columnwidth}\raggedright
\textcolor{red}{$\times$}\strut
\end{minipage}\tabularnewline
\begin{minipage}[t]{0.20\columnwidth}\raggedright
\texttt{ggpiestats} and \texttt{ggbarstats} (unpaired)\strut
\end{minipage} & \begin{minipage}[t]{0.16\columnwidth}\raggedright
Parametric\strut
\end{minipage} & \begin{minipage}[t]{0.31\columnwidth}\raggedright
\[\text{Pearson's}~ \chi^2 ~\text{test}\]\strut
\end{minipage} & \begin{minipage}[t]{0.12\columnwidth}\raggedright
Cramér's \emph{V}\strut
\end{minipage} & \begin{minipage}[t]{0.07\columnwidth}\raggedright
\textcolor{ForestGreen}{$\checkmark$}\strut
\end{minipage}\tabularnewline
\begin{minipage}[t]{0.20\columnwidth}\raggedright
\texttt{ggpiestats} and \texttt{ggbarstats} (paired)\strut
\end{minipage} & \begin{minipage}[t]{0.16\columnwidth}\raggedright
Parametric\strut
\end{minipage} & \begin{minipage}[t]{0.31\columnwidth}\raggedright
McNemar's test\strut
\end{minipage} & \begin{minipage}[t]{0.12\columnwidth}\raggedright
Cohen's \emph{g}\strut
\end{minipage} & \begin{minipage}[t]{0.07\columnwidth}\raggedright
\textcolor{ForestGreen}{$\checkmark$}\strut
\end{minipage}\tabularnewline
\begin{minipage}[t]{0.20\columnwidth}\raggedright
\texttt{ggpiestats}\strut
\end{minipage} & \begin{minipage}[t]{0.16\columnwidth}\raggedright
Parametric\strut
\end{minipage} & \begin{minipage}[t]{0.31\columnwidth}\raggedright
One-sample proportion test\strut
\end{minipage} & \begin{minipage}[t]{0.12\columnwidth}\raggedright
Cramér's \emph{V}\strut
\end{minipage} & \begin{minipage}[t]{0.07\columnwidth}\raggedright
\textcolor{ForestGreen}{$\checkmark$}\strut
\end{minipage}\tabularnewline
\begin{minipage}[t]{0.20\columnwidth}\raggedright
\texttt{ggscatterstats} and \texttt{ggcorrmat}\strut
\end{minipage} & \begin{minipage}[t]{0.16\columnwidth}\raggedright
Parametric\strut
\end{minipage} & \begin{minipage}[t]{0.31\columnwidth}\raggedright
Pearson's \emph{r}\strut
\end{minipage} & \begin{minipage}[t]{0.12\columnwidth}\raggedright
\emph{r}\strut
\end{minipage} & \begin{minipage}[t]{0.07\columnwidth}\raggedright
\textcolor{ForestGreen}{$\checkmark$}\strut
\end{minipage}\tabularnewline
\begin{minipage}[t]{0.20\columnwidth}\raggedright
\texttt{ggscatterstats} and \texttt{ggcorrmat}\strut
\end{minipage} & \begin{minipage}[t]{0.16\columnwidth}\raggedright
Non-parametric\strut
\end{minipage} & \begin{minipage}[t]{0.31\columnwidth}\raggedright
\[\text{Spearman's}~ \rho\]\strut
\end{minipage} & \begin{minipage}[t]{0.12\columnwidth}\raggedright
\[\rho\]\strut
\end{minipage} & \begin{minipage}[t]{0.07\columnwidth}\raggedright
\textcolor{ForestGreen}{$\checkmark$}\strut
\end{minipage}\tabularnewline
\begin{minipage}[t]{0.20\columnwidth}\raggedright
\texttt{ggscatterstats}and \texttt{ggcorrmat}\strut
\end{minipage} & \begin{minipage}[t]{0.16\columnwidth}\raggedright
Robust\strut
\end{minipage} & \begin{minipage}[t]{0.31\columnwidth}\raggedright
Percentage bend correlation\strut
\end{minipage} & \begin{minipage}[t]{0.12\columnwidth}\raggedright
\emph{r}\strut
\end{minipage} & \begin{minipage}[t]{0.07\columnwidth}\raggedright
\textcolor{ForestGreen}{$\checkmark$}\strut
\end{minipage}\tabularnewline
\begin{minipage}[t]{0.20\columnwidth}\raggedright
\texttt{gghistostats} and \texttt{ggdotplotstats}\strut
\end{minipage} & \begin{minipage}[t]{0.16\columnwidth}\raggedright
Parametric\strut
\end{minipage} & \begin{minipage}[t]{0.31\columnwidth}\raggedright
One-sample \emph{t}-test\strut
\end{minipage} & \begin{minipage}[t]{0.12\columnwidth}\raggedright
Cohen's \emph{d}, Hedge's \emph{g}\strut
\end{minipage} & \begin{minipage}[t]{0.07\columnwidth}\raggedright
\textcolor{ForestGreen}{$\checkmark$}\strut
\end{minipage}\tabularnewline
\begin{minipage}[t]{0.20\columnwidth}\raggedright
\texttt{gghistostats}\strut
\end{minipage} & \begin{minipage}[t]{0.16\columnwidth}\raggedright
Non-parametric\strut
\end{minipage} & \begin{minipage}[t]{0.31\columnwidth}\raggedright
One-sample Wilcoxon signed rank test\strut
\end{minipage} & \begin{minipage}[t]{0.12\columnwidth}\raggedright
\emph{r}\strut
\end{minipage} & \begin{minipage}[t]{0.07\columnwidth}\raggedright
\textcolor{ForestGreen}{$\checkmark$}\strut
\end{minipage}\tabularnewline
\begin{minipage}[t]{0.20\columnwidth}\raggedright
\texttt{gghistostats} and \texttt{ggdotplotstats}\strut
\end{minipage} & \begin{minipage}[t]{0.16\columnwidth}\raggedright
Robust\strut
\end{minipage} & \begin{minipage}[t]{0.31\columnwidth}\raggedright
One-sample percentile bootstrap\strut
\end{minipage} & \begin{minipage}[t]{0.12\columnwidth}\raggedright
robust estimator\strut
\end{minipage} & \begin{minipage}[t]{0.07\columnwidth}\raggedright
\textcolor{ForestGreen}{$\checkmark$}\strut
\end{minipage}\tabularnewline
\begin{minipage}[t]{0.20\columnwidth}\raggedright
\texttt{ggcoefstats}\strut
\end{minipage} & \begin{minipage}[t]{0.16\columnwidth}\raggedright
Parametric\strut
\end{minipage} & \begin{minipage}[t]{0.31\columnwidth}\raggedright
Regression models\strut
\end{minipage} & \begin{minipage}[t]{0.12\columnwidth}\raggedright
\[\beta\]\strut
\end{minipage} & \begin{minipage}[t]{0.07\columnwidth}\raggedright
\textcolor{ForestGreen}{$\checkmark$}\strut
\end{minipage}\tabularnewline
\bottomrule
\end{longtable}

\hypertarget{dealing-with-null-results}{%
\subsection{\texorpdfstring{Dealing with \textbf{null results}:}{Dealing with null results:}}\label{dealing-with-null-results}}

All functions therefore by default return Bayes Factor
in favor of the null hypothesis by default. If the null hypothesis can't be
rejected with the null hypothesis significance testing (NHST) approach, the
Bayesian approach can help index evidence in favor of the null hypothesis (i.e.,
\(BF\_{01}\)). By default, natural logarithms are shown because Bayes Factor
values can sometimes be pretty large. Having values on logarithmic scale also
makes it easy to compare evidence in favor alternative (\(BF\_{10}\)) versus null
(\(BF\_{01}\)) hypotheses (since \(log\_{e}(BF\_{01}) = - log\_{e}(BF\_{01})\)).

\hypertarget{avoiding-the-p-value-error}{%
\subsection{\texorpdfstring{Avoiding the \textbf{``p-value error''}:}{Avoiding the ``p-value error'':}}\label{avoiding-the-p-value-error}}

The \emph{p}-value indexes the probability that the researchers have falsely rejected
a true null hypothesis (Type I error, i.e.) and can rarely be \emph{exactly} 0. And
yet over 97,000 manuscripts on Google Scholar report the \emph{p}-value to be \texttt{p\ =0.000},
putatively due to relying on default computer software outputs
(Lilienfeld et al., \protect\hyperlink{ref-lilienfeldFiftyPsychologicalPsychiatric2015}{2015}). All \emph{p}-values displayed in
\texttt{ggstatsplot} plots avoid this mistake. Anything less than \texttt{p\ \textless{}\ 0.001} is
displayed as such (e.g, Figure \ref{fig:fig1}). The package deems it unimportant how
infinitesimally small the \emph{p}-values are and, instead, puts emphasis on the
effect size magnitudes and their 95\% CIs.

\newpage

\hypertarget{short-tutorial-on-primary-functions}{%
\section{Short tutorial on primary functions}\label{short-tutorial-on-primary-functions}}

Here are examples of the main functions currently supported in \texttt{ggstatsplot}.

\hypertarget{ggbetweenstats}{%
\subsection{\texorpdfstring{\texttt{ggbetweenstats}}{ggbetweenstats}}\label{ggbetweenstats}}

This function creates either a violin plot, a box plot, or a mix of two for
\textbf{between}-group or \textbf{between}-condition comparisons with results from
statistical tests in the subtitle. The simplest function call looks like this-

\begin{Shaded}
\begin{Highlighting}[]
\CommentTok{# loading needed libraries}
\KeywordTok{library}\NormalTok{(ggstatsplot)}

\CommentTok{# for reproducibility}
\KeywordTok{set.seed}\NormalTok{(}\DecValTok{123}\NormalTok{)}

\CommentTok{# plot}
\NormalTok{ggstatsplot}\OperatorTok{::}\KeywordTok{ggbetweenstats}\NormalTok{(}
  \DataTypeTok{data =}\NormalTok{ iris,}
  \DataTypeTok{x =}\NormalTok{ Species,}
  \DataTypeTok{y =}\NormalTok{ Sepal.Length,}
  \DataTypeTok{messages =} \OtherTok{FALSE}
\NormalTok{) }\OperatorTok{+}\StringTok{ }\CommentTok{# further modification outside of ggstatsplot}
\StringTok{  }\NormalTok{ggplot2}\OperatorTok{::}\KeywordTok{coord_cartesian}\NormalTok{(}\DataTypeTok{ylim =} \KeywordTok{c}\NormalTok{(}\DecValTok{3}\NormalTok{, }\DecValTok{8}\NormalTok{)) }\OperatorTok{+}
\StringTok{  }\NormalTok{ggplot2}\OperatorTok{::}\KeywordTok{scale_y_continuous}\NormalTok{(}\DataTypeTok{breaks =} \KeywordTok{seq}\NormalTok{(}\DecValTok{3}\NormalTok{, }\DecValTok{8}\NormalTok{, }\DataTypeTok{by =} \DecValTok{1}\NormalTok{))}
\end{Highlighting}
\end{Shaded}

\includegraphics[width=1\linewidth]{./figures/paper-ggbetweenstats1-1}

Note that this function returns object of class \texttt{ggplot} and thus can be further
modified using \texttt{ggplot2} functions.

A number of other arguments can be specified to make this plot even more
informative or change some of the default options. Additionally, this time we
will use a grouping variable that has only two levels. The function will
automatically switch from carrying out an ANOVA analysis to a \emph{t}-test.

\begin{Shaded}
\begin{Highlighting}[]
\CommentTok{# for reproducibility}
\KeywordTok{set.seed}\NormalTok{(}\DecValTok{123}\NormalTok{)}
\KeywordTok{library}\NormalTok{(ggplot2)}

\CommentTok{# let's leave out one of the factor levels and see if instead of anova, a t-test will be run}
\NormalTok{iris2 <-}\StringTok{ }\NormalTok{dplyr}\OperatorTok{::}\KeywordTok{filter}\NormalTok{(}\DataTypeTok{.data =}\NormalTok{ iris, Species }\OperatorTok{!=}\StringTok{ "setosa"}\NormalTok{)}

\CommentTok{# let's change the levels of our factors, a common routine in data analysis}
\CommentTok{# pipeline, to see if this function respects the new factor levels}
\NormalTok{iris2}\OperatorTok{$}\NormalTok{Species <-}\StringTok{ }\KeywordTok{factor}\NormalTok{(}\DataTypeTok{x =}\NormalTok{ iris2}\OperatorTok{$}\NormalTok{Species, }\DataTypeTok{levels =} \KeywordTok{c}\NormalTok{(}\StringTok{"virginica"}\NormalTok{, }\StringTok{"versicolor"}\NormalTok{))}

\CommentTok{# plot}
\NormalTok{ggstatsplot}\OperatorTok{::}\KeywordTok{ggbetweenstats}\NormalTok{(}
  \DataTypeTok{data =}\NormalTok{ iris2,}
  \DataTypeTok{x =}\NormalTok{ Species,}
  \DataTypeTok{y =}\NormalTok{ Sepal.Length,}
  \DataTypeTok{notch =} \OtherTok{TRUE}\NormalTok{, }\CommentTok{# show notched box plot}
  \DataTypeTok{mean.plotting =} \OtherTok{TRUE}\NormalTok{, }\CommentTok{# whether mean for each group is to be displayed}
  \DataTypeTok{mean.ci =} \OtherTok{TRUE}\NormalTok{, }\CommentTok{# whether to display confidence interval for means}
  \DataTypeTok{mean.label.size =} \FloatTok{2.5}\NormalTok{, }\CommentTok{# size of the label for mean}
  \DataTypeTok{type =} \StringTok{"parametric"}\NormalTok{, }\CommentTok{# which type of test is to be run}
  \DataTypeTok{k =} \DecValTok{3}\NormalTok{, }\CommentTok{# number of decimal places for statistical results}
  \DataTypeTok{outlier.tagging =} \OtherTok{TRUE}\NormalTok{, }\CommentTok{# whether outliers need to be tagged}
  \DataTypeTok{outlier.label =}\NormalTok{ Sepal.Width, }\CommentTok{# variable to be used for the outlier tag}
  \DataTypeTok{outlier.label.color =} \StringTok{"darkgreen"}\NormalTok{, }\CommentTok{# changing the color for the text label}
  \DataTypeTok{xlab =} \StringTok{"Type of Species"}\NormalTok{, }\CommentTok{# label for the x-axis variable}
  \DataTypeTok{ylab =} \StringTok{"Attribute: Sepal Length"}\NormalTok{, }\CommentTok{# label for the y-axis variable}
  \DataTypeTok{title =} \StringTok{"Dataset: Iris flower data set"}\NormalTok{, }\CommentTok{# title text for the plot}
  \DataTypeTok{ggtheme =}\NormalTok{ ggthemes}\OperatorTok{::}\KeywordTok{theme_fivethirtyeight}\NormalTok{(), }\CommentTok{# choosing a different theme}
  \DataTypeTok{ggstatsplot.layer =} \OtherTok{FALSE}\NormalTok{, }\CommentTok{# turn off ggstatsplot theme layer}
  \DataTypeTok{package =} \StringTok{"wesanderson"}\NormalTok{, }\CommentTok{# package from which color palette is to be taken}
  \DataTypeTok{palette =} \StringTok{"Darjeeling1"}\NormalTok{, }\CommentTok{# choosing a different color palette}
  \DataTypeTok{messages =} \OtherTok{FALSE}
\NormalTok{)}
\end{Highlighting}
\end{Shaded}

\includegraphics[width=1\linewidth]{./figures/paper-ggbetweenstats2-1}

Additionally, there is also a \texttt{grouped\_} variant of this function that makes it
easy to repeat the same operation across a \textbf{single} grouping variable:

\begin{Shaded}
\begin{Highlighting}[]
\CommentTok{# for reproducibility}
\KeywordTok{set.seed}\NormalTok{(}\DecValTok{123}\NormalTok{)}

\CommentTok{# plot}
\NormalTok{ggstatsplot}\OperatorTok{::}\KeywordTok{grouped_ggbetweenstats}\NormalTok{(}
  \DataTypeTok{data =}\NormalTok{ dplyr}\OperatorTok{::}\KeywordTok{filter}\NormalTok{(}
    \DataTypeTok{.data =}\NormalTok{ ggstatsplot}\OperatorTok{::}\NormalTok{movies_long,}
\NormalTok{    genre }\OperatorTok\StringTok{ }\KeywordTok{c}\NormalTok{(}\StringTok{"Action"}\NormalTok{, }\StringTok{"Action Comedy"}\NormalTok{, }\StringTok{"Action Drama"}\NormalTok{, }\StringTok{"Comedy"}\NormalTok{)}
\NormalTok{  ),}
  \DataTypeTok{x =}\NormalTok{ mpaa,}
  \DataTypeTok{y =}\NormalTok{ length,}
  \DataTypeTok{grouping.var =}\NormalTok{ genre, }\CommentTok{# grouping variable}
  \DataTypeTok{pairwise.comparisons =} \OtherTok{TRUE}\NormalTok{, }\CommentTok{# display significant pairwise comparisons}
  \DataTypeTok{pairwise.annotation =} \StringTok{"p.value"}\NormalTok{, }\CommentTok{# how do you want to annotate the pairwise comparisons}
  \DataTypeTok{p.adjust.method =} \StringTok{"bonferroni"}\NormalTok{, }\CommentTok{# method for adjusting p-values for multiple comparisons}
  \DataTypeTok{conf.level =} \FloatTok{0.99}\NormalTok{, }\CommentTok{# changing confidence level to 99%}
  \DataTypeTok{ggplot.component =} \KeywordTok{list}\NormalTok{( }\CommentTok{# adding new components to `ggstatsplot` default}
\NormalTok{    ggplot2}\OperatorTok{::}\KeywordTok{scale_y_continuous}\NormalTok{(}\DataTypeTok{sec.axis =}\NormalTok{ ggplot2}\OperatorTok{::}\KeywordTok{dup_axis}\NormalTok{())}
\NormalTok{  ),}
  \DataTypeTok{k =} \DecValTok{3}\NormalTok{,}
  \DataTypeTok{title.prefix =} \StringTok{"Movie genre"}\NormalTok{,}
  \DataTypeTok{caption =} \KeywordTok{substitute}\NormalTok{(}\KeywordTok{paste}\NormalTok{(}\KeywordTok{italic}\NormalTok{(}\StringTok{"Source"}\NormalTok{), }\StringTok{":IMDb (Internet Movie Database)"}\NormalTok{)),}
  \DataTypeTok{palette =} \StringTok{"default_jama"}\NormalTok{,}
  \DataTypeTok{package =} \StringTok{"ggsci"}\NormalTok{,}
  \DataTypeTok{messages =} \OtherTok{FALSE}\NormalTok{,}
  \DataTypeTok{nrow =} \DecValTok{2}\NormalTok{,}
  \DataTypeTok{title.text =} \StringTok{"Differences in movie length by mpaa ratings for different genres"}
\NormalTok{)}
\end{Highlighting}
\end{Shaded}

\includegraphics[width=1\linewidth]{./figures/paper-ggbetweenstats3-1}

\hypertarget{summary-of-tests}{%
\subsubsection{Summary of tests}\label{summary-of-tests}}

Following (between-subjects) tests are carried out for each type of analyses-

\begin{longtable}[]{@{}lll@{}}
\toprule
Type & No.~of groups & Test\tabularnewline
\midrule
\endhead
Parametric & \textgreater{} 2 & Fisher's or Welch's one-way ANOVA\tabularnewline
Non-parametric & \textgreater{} 2 & Kruskal--Wallis one-way ANOVA\tabularnewline
Robust & \textgreater{} 2 & Heteroscedastic one-way ANOVA for trimmed means\tabularnewline
Bayes Factor & \textgreater{} 2 & Fisher's ANOVA\tabularnewline
Parametric & 2 & Student's or Welch's \emph{t}-test\tabularnewline
Non-parametric & 2 & Mann--Whitney \emph{U} test\tabularnewline
Robust & 2 & Yuen's test for trimmed means\tabularnewline
Bayes Factor & 2 & Student's \emph{t}-test\tabularnewline
\bottomrule
\end{longtable}

The omnibus effect in one-way ANOVA design can also be followed up with more
focal pairwise comparison tests. Here is a summary of \emph{multiple pairwise
comparison} tests supported in \emph{ggbetweenstats}-

\begin{longtable}[]{@{}llll@{}}
\toprule
Type & Equal variance? & Test & \emph{p}-value adjustment?\tabularnewline
\midrule
\endhead
Parametric & No & Games-Howell test & \textcolor{ForestGreen}{Yes}\tabularnewline
Parametric & Yes & Student's \emph{t}-test & \textcolor{ForestGreen}{Yes}\tabularnewline
Non-parametric & No & Dwass-Steel-Crichtlow-Fligner test & \textcolor{ForestGreen}{Yes}\tabularnewline
Robust & No & Yuen's trimmed means test & \textcolor{ForestGreen}{Yes}\tabularnewline
Bayes Factor & No & \textcolor{red}{No} & \textcolor{red}{No}\tabularnewline
Bayes Factor & Yes & \textcolor{red}{No} & \textcolor{red}{No}\tabularnewline
\bottomrule
\end{longtable}

For more, see the \texttt{ggbetweenstats} vignette:
\url{https://indrajeetpatil.github.io/ggstatsplot/articles/web_only/ggbetweenstats.html}

\hypertarget{ggwithinstats}{%
\subsection{\texorpdfstring{\texttt{ggwithinstats}}{ggwithinstats}}\label{ggwithinstats}}

\texttt{ggbetweenstats} function has an identical twin function \texttt{ggwithinstats} for
repeated measures designs that behaves in the same fashion with a few minor
tweaks introduced to properly visualize the repeated measures design. As can be
seen from an example below, the only difference between the plot structure is
that now the group means are connected by paths to highlight the fact that these
data are paired with each other.

\begin{Shaded}
\begin{Highlighting}[]
\CommentTok{# for reproducibility and data}
\KeywordTok{set.seed}\NormalTok{(}\DecValTok{123}\NormalTok{)}
\KeywordTok{library}\NormalTok{(WRS2)}

\CommentTok{# plot}
\NormalTok{ggstatsplot}\OperatorTok{::}\KeywordTok{ggwithinstats}\NormalTok{(}
  \DataTypeTok{data =}\NormalTok{ WRS2}\OperatorTok{::}\NormalTok{WineTasting,}
  \DataTypeTok{x =}\NormalTok{ Wine,}
  \DataTypeTok{y =}\NormalTok{ Taste,}
  \DataTypeTok{sort =} \StringTok{"descending"}\NormalTok{, }\CommentTok{# ordering groups along the x-axis based on}
  \DataTypeTok{sort.fun =}\NormalTok{ median, }\CommentTok{# values of `y` variable}
  \DataTypeTok{pairwise.comparisons =} \OtherTok{TRUE}\NormalTok{,}
  \DataTypeTok{pairwise.display =} \StringTok{"s"}\NormalTok{,}
  \DataTypeTok{pairwise.annotation =} \StringTok{"p"}\NormalTok{,}
  \DataTypeTok{title =} \StringTok{"Wine tasting"}\NormalTok{,}
  \DataTypeTok{caption =} \StringTok{"Data from: `WRS2` R package"}\NormalTok{,}
  \DataTypeTok{ggtheme =}\NormalTok{ ggthemes}\OperatorTok{::}\KeywordTok{theme_fivethirtyeight}\NormalTok{(),}
  \DataTypeTok{ggstatsplot.layer =} \OtherTok{FALSE}\NormalTok{,}
  \DataTypeTok{messages =} \OtherTok{FALSE}
\NormalTok{)}
\end{Highlighting}
\end{Shaded}

\includegraphics[width=1\linewidth]{./figures/paper-ggwithinstats1-1}

As with the \texttt{ggbetweenstats}, this function also has a \texttt{grouped\_} variant that
makes repeating the same analysis across a single grouping variable quicker.

\begin{Shaded}
\begin{Highlighting}[]
\CommentTok{# common setup}
\KeywordTok{set.seed}\NormalTok{(}\DecValTok{123}\NormalTok{)}

\CommentTok{# getting data in tidy format}
\NormalTok{data_bugs <-}\StringTok{ }\NormalTok{ggstatsplot}\OperatorTok{::}\NormalTok{bugs_long }\OperatorTok
\StringTok{  }\NormalTok{dplyr}\OperatorTok{::}\KeywordTok{filter}\NormalTok{(}\DataTypeTok{.data =}\NormalTok{ ., region }\OperatorTok\StringTok{ }\KeywordTok{c}\NormalTok{(}\StringTok{"Europe"}\NormalTok{, }\StringTok{"North America"}\NormalTok{))}

\CommentTok{# plot}
\NormalTok{ggstatsplot}\OperatorTok{::}\KeywordTok{grouped_ggwithinstats}\NormalTok{(}
  \DataTypeTok{data =}\NormalTok{ dplyr}\OperatorTok{::}\KeywordTok{filter}\NormalTok{(data_bugs, condition }\OperatorTok\StringTok{ }\KeywordTok{c}\NormalTok{(}\StringTok{"LDLF"}\NormalTok{, }\StringTok{"LDHF"}\NormalTok{)),}
  \DataTypeTok{x =}\NormalTok{ condition,}
  \DataTypeTok{y =}\NormalTok{ desire,}
  \DataTypeTok{xlab =} \StringTok{"Condition"}\NormalTok{,}
  \DataTypeTok{ylab =} \StringTok{"Desire to kill an artrhopod"}\NormalTok{,}
  \DataTypeTok{grouping.var =}\NormalTok{ region,}
  \DataTypeTok{outlier.tagging =} \OtherTok{TRUE}\NormalTok{,}
  \DataTypeTok{outlier.label =}\NormalTok{ education,}
  \DataTypeTok{ggtheme =}\NormalTok{ ggthemes}\OperatorTok{::}\KeywordTok{theme_fivethirtyeight}\NormalTok{(),}
  \DataTypeTok{ggstatsplot.layer =} \OtherTok{FALSE}\NormalTok{,}
  \DataTypeTok{messages =} \OtherTok{FALSE}
\NormalTok{)}
\end{Highlighting}
\end{Shaded}

\includegraphics[width=1\linewidth]{./figures/paper-ggwithinstats2-1}

\hypertarget{summary-of-tests-1}{%
\subsubsection{Summary of tests}\label{summary-of-tests-1}}

Following (within-subjects) tests are carried out for each type of analyses-

\begin{longtable}[]{@{}lll@{}}
\toprule
\begin{minipage}[b]{0.26\columnwidth}\raggedright
Type\strut
\end{minipage} & \begin{minipage}[b]{0.07\columnwidth}\raggedright
No.~of groups\strut
\end{minipage} & \begin{minipage}[b]{0.59\columnwidth}\raggedright
Test\strut
\end{minipage}\tabularnewline
\midrule
\endhead
\begin{minipage}[t]{0.26\columnwidth}\raggedright
Parametric\strut
\end{minipage} & \begin{minipage}[t]{0.07\columnwidth}\raggedright
\textgreater{} 2\strut
\end{minipage} & \begin{minipage}[t]{0.59\columnwidth}\raggedright
One-way repeated measures ANOVA\strut
\end{minipage}\tabularnewline
\begin{minipage}[t]{0.26\columnwidth}\raggedright
Non-parametric\strut
\end{minipage} & \begin{minipage}[t]{0.07\columnwidth}\raggedright
\textgreater{} 2\strut
\end{minipage} & \begin{minipage}[t]{0.59\columnwidth}\raggedright
Friedman test\strut
\end{minipage}\tabularnewline
\begin{minipage}[t]{0.26\columnwidth}\raggedright
Robust\strut
\end{minipage} & \begin{minipage}[t]{0.07\columnwidth}\raggedright
\textgreater{} 2\strut
\end{minipage} & \begin{minipage}[t]{0.59\columnwidth}\raggedright
Heteroscedastic one-way repeated measures ANOVA for trimmed means\strut
\end{minipage}\tabularnewline
\begin{minipage}[t]{0.26\columnwidth}\raggedright
Bayes Factor\strut
\end{minipage} & \begin{minipage}[t]{0.07\columnwidth}\raggedright
\textgreater{} 2\strut
\end{minipage} & \begin{minipage}[t]{0.59\columnwidth}\raggedright
One-way repeated measures ANOVA\strut
\end{minipage}\tabularnewline
\begin{minipage}[t]{0.26\columnwidth}\raggedright
Parametric\strut
\end{minipage} & \begin{minipage}[t]{0.07\columnwidth}\raggedright
2\strut
\end{minipage} & \begin{minipage}[t]{0.59\columnwidth}\raggedright
Student's \emph{t}-test\strut
\end{minipage}\tabularnewline
\begin{minipage}[t]{0.26\columnwidth}\raggedright
Non-parametric\strut
\end{minipage} & \begin{minipage}[t]{0.07\columnwidth}\raggedright
2\strut
\end{minipage} & \begin{minipage}[t]{0.59\columnwidth}\raggedright
Wilcoxon signed-rank test\strut
\end{minipage}\tabularnewline
\begin{minipage}[t]{0.26\columnwidth}\raggedright
Robust\strut
\end{minipage} & \begin{minipage}[t]{0.07\columnwidth}\raggedright
2\strut
\end{minipage} & \begin{minipage}[t]{0.59\columnwidth}\raggedright
Yuen's test on trimmed means for dependent samples\strut
\end{minipage}\tabularnewline
\begin{minipage}[t]{0.26\columnwidth}\raggedright
Bayes Factor\strut
\end{minipage} & \begin{minipage}[t]{0.07\columnwidth}\raggedright
2\strut
\end{minipage} & \begin{minipage}[t]{0.59\columnwidth}\raggedright
Student's \emph{t}-test\strut
\end{minipage}\tabularnewline
\bottomrule
\end{longtable}

The omnibus effect in one-way ANOVA design can also be followed up with more
focal pairwise comparison tests. Here is a summary of \emph{multiple pairwise
comparison} tests supported in \emph{ggwithinstats}-

\begin{longtable}[]{@{}lll@{}}
\toprule
Type & Test & \emph{p}-value adjustment?\tabularnewline
\midrule
\endhead
Parametric & Student's \emph{t}-test & \textcolor{ForestGreen}{Yes}\tabularnewline
Non-parametric & Durbin-Conover test & \textcolor{ForestGreen}{Yes}\tabularnewline
Robust & Yuen's trimmed means test & \textcolor{ForestGreen}{Yes}\tabularnewline
Bayes Factor & \textcolor{red}{No} & \textcolor{red}{No}\tabularnewline
\bottomrule
\end{longtable}

For more, see the \texttt{ggwithinstats} vignette:
\url{https://indrajeetpatil.github.io/ggstatsplot/articles/web_only/ggwithinstats.html}

\hypertarget{ggscatterstats}{%
\subsection{\texorpdfstring{\texttt{ggscatterstats}}{ggscatterstats}}\label{ggscatterstats}}

This function creates a scatterplot with marginal distributions overlaid on the
axes (from \texttt{ggExtra::ggMarginal}) and results from statistical tests in the
subtitle:

\begin{Shaded}
\begin{Highlighting}[]
\NormalTok{ggstatsplot}\OperatorTok{::}\KeywordTok{ggscatterstats}\NormalTok{(}
  \DataTypeTok{data =}\NormalTok{ ggplot2}\OperatorTok{::}\NormalTok{msleep,}
  \DataTypeTok{x =}\NormalTok{ sleep_rem,}
  \DataTypeTok{y =}\NormalTok{ awake,}
  \DataTypeTok{xlab =} \StringTok{"REM sleep (in hours)"}\NormalTok{,}
  \DataTypeTok{ylab =} \StringTok{"Amount of time spent awake (in hours)"}\NormalTok{,}
  \DataTypeTok{title =} \StringTok{"Understanding mammalian sleep"}\NormalTok{,}
  \DataTypeTok{messages =} \OtherTok{FALSE}
\NormalTok{)}
\end{Highlighting}
\end{Shaded}

\includegraphics[width=1\linewidth]{./figures/paper-ggscatterstats1-1}

The available marginal distributions are-

\begin{itemize}
\tightlist
\item
  histograms
\item
  boxplots
\item
  density
\item
  violin
\item
  densigram (density + histogram)
\end{itemize}

Number of other arguments can be specified to modify this basic plot-

\begin{Shaded}
\begin{Highlighting}[]
\CommentTok{# for reproducibility}
\KeywordTok{set.seed}\NormalTok{(}\DecValTok{123}\NormalTok{)}

\CommentTok{# plot}
\NormalTok{ggstatsplot}\OperatorTok{::}\KeywordTok{ggscatterstats}\NormalTok{(}
  \DataTypeTok{data =}\NormalTok{ dplyr}\OperatorTok{::}\KeywordTok{filter}\NormalTok{(}\DataTypeTok{.data =}\NormalTok{ ggstatsplot}\OperatorTok{::}\NormalTok{movies_long, genre }\OperatorTok{==}\StringTok{ "Action"}\NormalTok{),}
  \DataTypeTok{x =}\NormalTok{ budget,}
  \DataTypeTok{y =}\NormalTok{ rating,}
  \DataTypeTok{type =} \StringTok{"robust"}\NormalTok{, }\CommentTok{# type of test that needs to be run}
  \DataTypeTok{conf.level =} \FloatTok{0.99}\NormalTok{, }\CommentTok{# confidence level}
  \DataTypeTok{xlab =} \StringTok{"Movie budget (in million/ US$)"}\NormalTok{, }\CommentTok{# label for x axis}
  \DataTypeTok{ylab =} \StringTok{"IMDB rating"}\NormalTok{, }\CommentTok{# label for y axis}
  \DataTypeTok{label.var =} \StringTok{"title"}\NormalTok{, }\CommentTok{# variable for labeling data points}
  \DataTypeTok{label.expression =} \StringTok{"rating < 5 & budget > 100"}\NormalTok{, }\CommentTok{# expression that decides which points to label}
  \DataTypeTok{line.color =} \StringTok{"yellow"}\NormalTok{, }\CommentTok{# changing regression line color line}
  \DataTypeTok{title =} \StringTok{"Movie budget and IMDB rating (action)"}\NormalTok{, }\CommentTok{# title text for the plot}
  \DataTypeTok{caption =} \KeywordTok{expression}\NormalTok{( }\CommentTok{# caption text for the plot}
    \KeywordTok{paste}\NormalTok{(}\KeywordTok{italic}\NormalTok{(}\StringTok{"Note"}\NormalTok{), }\StringTok{": IMDB stands for Internet Movie DataBase"}\NormalTok{)}
\NormalTok{  ),}
  \DataTypeTok{ggtheme =}\NormalTok{ hrbrthemes}\OperatorTok{::}\KeywordTok{theme_ipsum_ps}\NormalTok{(), }\CommentTok{# choosing a different theme}
  \DataTypeTok{ggstatsplot.layer =} \OtherTok{FALSE}\NormalTok{, }\CommentTok{# turn off ggstatsplot theme layer}
  \DataTypeTok{marginal.type =} \StringTok{"density"}\NormalTok{, }\CommentTok{# type of marginal distribution to be displayed}
  \DataTypeTok{xfill =} \StringTok{"#0072B2"}\NormalTok{, }\CommentTok{# color fill for x-axis marginal distribution}
  \DataTypeTok{yfill =} \StringTok{"#009E73"}\NormalTok{, }\CommentTok{# color fill for y-axis marginal distribution}
  \DataTypeTok{xalpha =} \FloatTok{0.6}\NormalTok{, }\CommentTok{# transparency for x-axis marginal distribution}
  \DataTypeTok{yalpha =} \FloatTok{0.6}\NormalTok{, }\CommentTok{# transparency for y-axis marginal distribution}
  \DataTypeTok{centrality.para =} \StringTok{"median"}\NormalTok{, }\CommentTok{# central tendency lines to be displayed}
  \DataTypeTok{messages =} \OtherTok{FALSE} \CommentTok{# turn off messages and notes}
\NormalTok{)}
\end{Highlighting}
\end{Shaded}

\includegraphics[width=1\linewidth]{./figures/paper-ggscatterstats2-1}

Additionally, there is also a \texttt{grouped\_} variant of this function that makes it
easy to repeat the same operation across a \textbf{single} grouping variable. Also,
note that, as opposed to the other functions, this function does not return a
\texttt{ggplot} object and any modification you want to make can be made in advance
using \texttt{ggplot.component} argument (available for all functions, but especially
useful for this particular function):

\begin{Shaded}
\begin{Highlighting}[]
\CommentTok{# for reproducibility}
\KeywordTok{set.seed}\NormalTok{(}\DecValTok{123}\NormalTok{)}

\CommentTok{# plot}
\NormalTok{ggstatsplot}\OperatorTok{::}\KeywordTok{grouped_ggscatterstats}\NormalTok{(}
  \DataTypeTok{data =}\NormalTok{ dplyr}\OperatorTok{::}\KeywordTok{filter}\NormalTok{(}
    \DataTypeTok{.data =}\NormalTok{ ggstatsplot}\OperatorTok{::}\NormalTok{movies_long,}
\NormalTok{    genre }\OperatorTok\StringTok{ }\KeywordTok{c}\NormalTok{(}\StringTok{"Action"}\NormalTok{, }\StringTok{"Action Comedy"}\NormalTok{, }\StringTok{"Action Drama"}\NormalTok{, }\StringTok{"Comedy"}\NormalTok{)}
\NormalTok{  ),}
  \DataTypeTok{x =}\NormalTok{ rating,}
  \DataTypeTok{y =}\NormalTok{ length,}
  \DataTypeTok{label.var =}\NormalTok{ title,}
  \DataTypeTok{label.expression =}\NormalTok{ length }\OperatorTok{>}\StringTok{ }\DecValTok{200}\NormalTok{,}
  \DataTypeTok{conf.level =} \FloatTok{0.99}\NormalTok{,}
  \DataTypeTok{k =} \DecValTok{3}\NormalTok{, }\CommentTok{# no. of decimal places in the results}
  \DataTypeTok{xfill =} \StringTok{"#E69F00"}\NormalTok{,}
  \DataTypeTok{yfill =} \StringTok{"#8b3058"}\NormalTok{,}
  \DataTypeTok{xlab =} \StringTok{"IMDB rating"}\NormalTok{,}
  \DataTypeTok{grouping.var =}\NormalTok{ genre, }\CommentTok{# grouping variable}
  \DataTypeTok{title.prefix =} \StringTok{"Movie genre"}\NormalTok{,}
  \DataTypeTok{ggtheme =}\NormalTok{ ggplot2}\OperatorTok{::}\KeywordTok{theme_grey}\NormalTok{(),}
  \DataTypeTok{ggplot.component =} \KeywordTok{list}\NormalTok{(}
\NormalTok{    ggplot2}\OperatorTok{::}\KeywordTok{scale_x_continuous}\NormalTok{(}\DataTypeTok{breaks =} \KeywordTok{seq}\NormalTok{(}\DecValTok{2}\NormalTok{, }\DecValTok{9}\NormalTok{, }\DecValTok{1}\NormalTok{), }\DataTypeTok{limits =}\NormalTok{ (}\KeywordTok{c}\NormalTok{(}\DecValTok{2}\NormalTok{, }\DecValTok{9}\NormalTok{)))}
\NormalTok{  ),}
  \DataTypeTok{messages =} \OtherTok{FALSE}\NormalTok{,}
  \DataTypeTok{nrow =} \DecValTok{2}\NormalTok{,}
  \DataTypeTok{title.text =} \StringTok{"Relationship between movie length by IMDB ratings for different genres"}
\NormalTok{)}
\end{Highlighting}
\end{Shaded}

\includegraphics[width=1\linewidth]{./figures/paper-ggscatterstats3-1}

\hypertarget{summary-of-tests-2}{%
\subsubsection{Summary of tests}\label{summary-of-tests-2}}

Following tests are carried out for each type of analyses. Additionally, the
correlation coefficients (and their confidence intervals) are used as effect
sizes-

\begin{longtable}[]{@{}lll@{}}
\toprule
Type & Test & CI?\tabularnewline
\midrule
\endhead
Parametric & Pearson's correlation coefficient & \textcolor{ForestGreen}{Yes}\tabularnewline
Non-parametric & Spearman's rank correlation coefficient & \textcolor{ForestGreen}{Yes}\tabularnewline
Robust & Percentage bend correlation coefficient & \textcolor{ForestGreen}{Yes}\tabularnewline
Bayes Factor & Pearson's correlation coefficient & \textcolor{red}{No}\tabularnewline
\bottomrule
\end{longtable}

For more, see the \texttt{ggscatterstats} vignette:
\url{https://indrajeetpatil.github.io/ggstatsplot/articles/web_only/ggscatterstats.html}

\hypertarget{ggpiestats}{%
\subsection{\texorpdfstring{\texttt{ggpiestats}}{ggpiestats}}\label{ggpiestats}}

This function creates a pie chart for categorical or nominal variables with
results from contingency table analysis (Pearson's \(\chi^2\) test for
between-subjects design and McNemar's \(\chi^2\) test for within-subjects design)
included in the subtitle of the plot. If only one categorical variable is
entered, results from one-sample proportion test (i.e., a \(\chi^2\) goodness of
fit/gof test) will be displayed as a subtitle.

Here is an example of a case where the theoretical question is about proportions
for different levels of a single nominal variable:

\begin{Shaded}
\begin{Highlighting}[]
\CommentTok{# for reproducibility}
\KeywordTok{set.seed}\NormalTok{(}\DecValTok{123}\NormalTok{)}

\CommentTok{# plot}
\NormalTok{ggstatsplot}\OperatorTok{::}\KeywordTok{ggpiestats}\NormalTok{(}
  \DataTypeTok{data =}\NormalTok{ ggplot2}\OperatorTok{::}\NormalTok{msleep,}
  \DataTypeTok{x =}\NormalTok{ vore,}
  \DataTypeTok{title =} \StringTok{"Composition of vore types among mammals"}\NormalTok{,}
  \DataTypeTok{messages =} \OtherTok{FALSE}
\NormalTok{)}
\end{Highlighting}
\end{Shaded}

\includegraphics[width=1\linewidth]{./figures/paper-ggpiestats1-1}

This function can also be used to study an interaction between two categorical
variables:

\begin{Shaded}
\begin{Highlighting}[]
\CommentTok{# for reproducibility}
\KeywordTok{set.seed}\NormalTok{(}\DecValTok{123}\NormalTok{)}

\CommentTok{# plot}
\NormalTok{ggstatsplot}\OperatorTok{::}\KeywordTok{ggpiestats}\NormalTok{(}
  \DataTypeTok{data =}\NormalTok{ mtcars,}
  \DataTypeTok{x =}\NormalTok{ am,}
  \DataTypeTok{y =}\NormalTok{ cyl,}
  \DataTypeTok{conf.level =} \FloatTok{0.99}\NormalTok{, }\CommentTok{# confidence interval for effect size measure}
  \DataTypeTok{title =} \StringTok{"Dataset: Motor Trend Car Road Tests"}\NormalTok{, }\CommentTok{# title for the plot}
  \DataTypeTok{stat.title =} \StringTok{"interaction: "}\NormalTok{, }\CommentTok{# title for the results}
  \DataTypeTok{legend.title =} \StringTok{"Transmission"}\NormalTok{, }\CommentTok{# title for the legend}
  \DataTypeTok{factor.levels =} \KeywordTok{c}\NormalTok{(}\StringTok{"1 = manual"}\NormalTok{, }\StringTok{"0 = automatic"}\NormalTok{), }\CommentTok{# renaming the factor level names (`x`)}
  \DataTypeTok{facet.wrap.name =} \StringTok{"No. of cylinders"}\NormalTok{, }\CommentTok{# name for the facetting variable}
  \DataTypeTok{slice.label =} \StringTok{"counts"}\NormalTok{, }\CommentTok{# show counts data instead of percentages}
  \DataTypeTok{package =} \StringTok{"ggsci"}\NormalTok{, }\CommentTok{# package from which color palette is to be taken}
  \DataTypeTok{palette =} \StringTok{"default_jama"}\NormalTok{, }\CommentTok{# choosing a different color palette}
  \DataTypeTok{caption =} \KeywordTok{substitute}\NormalTok{( }\CommentTok{# text for the caption}
    \KeywordTok{paste}\NormalTok{(}\KeywordTok{italic}\NormalTok{(}\StringTok{"Source"}\NormalTok{), }\StringTok{": 1974 Motor Trend US magazine"}\NormalTok{)}
\NormalTok{  ),}
  \DataTypeTok{messages =} \OtherTok{FALSE} \CommentTok{# turn off messages and notes}
\NormalTok{)}
\end{Highlighting}
\end{Shaded}

\includegraphics[width=1\linewidth]{./figures/paper-ggpiestats2-1}

In case of repeated measures designs, setting \texttt{paired\ =\ TRUE} will produce
results from McNemar's \(\chi^2\) test-

\begin{Shaded}
\begin{Highlighting}[]
\CommentTok{# for reproducibility}
\KeywordTok{set.seed}\NormalTok{(}\DecValTok{123}\NormalTok{)}

\CommentTok{# data}
\NormalTok{survey.data <-}\StringTok{ }\KeywordTok{data.frame}\NormalTok{(}
  \StringTok{`}\DataTypeTok{1st survey}\StringTok{`}\NormalTok{ =}\StringTok{ }\KeywordTok{c}\NormalTok{(}\StringTok{"Approve"}\NormalTok{, }\StringTok{"Approve"}\NormalTok{, }\StringTok{"Disapprove"}\NormalTok{, }\StringTok{"Disapprove"}\NormalTok{),}
  \StringTok{`}\DataTypeTok{2nd survey}\StringTok{`}\NormalTok{ =}\StringTok{ }\KeywordTok{c}\NormalTok{(}\StringTok{"Approve"}\NormalTok{, }\StringTok{"Disapprove"}\NormalTok{, }\StringTok{"Approve"}\NormalTok{, }\StringTok{"Disapprove"}\NormalTok{),}
  \StringTok{`}\DataTypeTok{Counts}\StringTok{`}\NormalTok{ =}\StringTok{ }\KeywordTok{c}\NormalTok{(}\DecValTok{794}\NormalTok{, }\DecValTok{150}\NormalTok{, }\DecValTok{86}\NormalTok{, }\DecValTok{570}\NormalTok{),}
  \DataTypeTok{check.names =} \OtherTok{FALSE}
\NormalTok{)}

\CommentTok{# plot}
\NormalTok{ggstatsplot}\OperatorTok{::}\KeywordTok{ggpiestats}\NormalTok{(}
  \DataTypeTok{data =}\NormalTok{ survey.data,}
  \DataTypeTok{x =} \StringTok{`}\DataTypeTok{1st survey}\StringTok{`}\NormalTok{,}
  \DataTypeTok{y =} \StringTok{`}\DataTypeTok{2nd survey}\StringTok{`}\NormalTok{,}
  \DataTypeTok{counts =}\NormalTok{ Counts,}
  \DataTypeTok{paired =} \OtherTok{TRUE}\NormalTok{, }\CommentTok{# within-subjects design}
  \DataTypeTok{conf.level =} \FloatTok{0.99}\NormalTok{, }\CommentTok{# confidence interval for effect size measure}
  \DataTypeTok{package =} \StringTok{"wesanderson"}\NormalTok{,}
  \DataTypeTok{palette =} \StringTok{"Royal1"}
\NormalTok{)}
\CommentTok{#> Note: 99% CI for effect size estimate was computed with 100 bootstrap samples.}
\CommentTok{#> # A tibble: 2 x 11}
\CommentTok{#>   `2nd survey` counts  perc N         Approve Disapprove `Chi-squared`}
\CommentTok{#>   <fct>         <int> <dbl> <chr>     <chr>   <chr>              <dbl>}
\CommentTok{#> 1 Disapprove      720   45  (n = 720) 20.83%  79.17%              245 }
\CommentTok{#> 2 Approve         880   55. (n = 880) 90.23%  9.77%               570.}
\CommentTok{#>     p.value    df method                                   significance}
\CommentTok{#>       <dbl> <dbl> <chr>                                    <chr>       }
\CommentTok{#> 1 3.20e- 55     1 Chi-squared test for given probabilities ***         }
\CommentTok{#> 2 6.80e-126     1 Chi-squared test for given probabilities ***}
\end{Highlighting}
\end{Shaded}

\includegraphics[width=1\linewidth]{./figures/paper-ggpiestats3-1}

Note that when a two-way table is present (i.e., when both \texttt{x} and
\texttt{y} arguments are specified), \emph{p}-values for results from one-sample
proportion tests are displayed in each facet in the form of asterisks with the
following convention:

\begin{itemize}
\tightlist
\item
  \(***\): \(p < 0.001\)
\item
  \(**\): \(p < 0.01\)
\item
  \(*\): \(p < 0.05\)
\item
  \(ns\): \(p > 0.05\)
\end{itemize}

Additionally, there is also a \texttt{grouped\_} variant of this function that makes it
easy to repeat the same operation across a \textbf{single} grouping variable:

\begin{Shaded}
\begin{Highlighting}[]
\CommentTok{# for reproducibility}
\KeywordTok{set.seed}\NormalTok{(}\DecValTok{123}\NormalTok{)}

\CommentTok{# plot}
\NormalTok{ggstatsplot}\OperatorTok{::}\KeywordTok{grouped_ggpiestats}\NormalTok{(}
\NormalTok{  dplyr}\OperatorTok{::}\KeywordTok{filter}\NormalTok{(}
    \DataTypeTok{.data =}\NormalTok{ ggstatsplot}\OperatorTok{::}\NormalTok{movies_long,}
\NormalTok{    genre }\OperatorTok\StringTok{ }\KeywordTok{c}\NormalTok{(}\StringTok{"Action"}\NormalTok{, }\StringTok{"Action Comedy"}\NormalTok{, }\StringTok{"Action Drama"}\NormalTok{, }\StringTok{"Comedy"}\NormalTok{)}
\NormalTok{  ),}
  \DataTypeTok{x =}\NormalTok{ mpaa,}
  \DataTypeTok{grouping.var =}\NormalTok{ genre, }\CommentTok{# grouping variable}
  \DataTypeTok{title.prefix =} \StringTok{"Movie genre"}\NormalTok{, }\CommentTok{# prefix for the facetted title}
  \DataTypeTok{label.text.size =} \DecValTok{3}\NormalTok{, }\CommentTok{# text size for slice labels}
  \DataTypeTok{slice.label =} \StringTok{"both"}\NormalTok{, }\CommentTok{# show both counts and percentage data}
  \DataTypeTok{perc.k =} \DecValTok{1}\NormalTok{, }\CommentTok{# no. of decimal places for percentages}
  \DataTypeTok{palette =} \StringTok{"brightPastel"}\NormalTok{,}
  \DataTypeTok{package =} \StringTok{"quickpalette"}\NormalTok{,}
  \DataTypeTok{messages =} \OtherTok{FALSE}\NormalTok{,}
  \DataTypeTok{nrow =} \DecValTok{2}\NormalTok{,}
  \DataTypeTok{title.text =} \StringTok{"Composition of MPAA ratings for different genres"}
\NormalTok{)}
\end{Highlighting}
\end{Shaded}

\includegraphics[width=1\linewidth]{./figures/paper-ggpiestats4-1}

\hypertarget{summary-of-tests-3}{%
\subsubsection{Summary of tests}\label{summary-of-tests-3}}

Following tests are carried out for each type of analyses-

\begin{longtable}[]{@{}lll@{}}
\toprule
Type of data & Design & Test\tabularnewline
\midrule
\endhead
Unpaired & \(n \times p\) contingency table & Pearson's \(\chi^{2}\) test\tabularnewline
Paired & \(n \times p\) contingency table & McNemar's \(\chi^{2}\) test\tabularnewline
Frequency & \(n \times 1\) contingency table & Goodness of fit (\(\chi^{2}\))\tabularnewline
\bottomrule
\end{longtable}

Following effect sizes (and confidence intervals/CI) are available for each type
of test-

\begin{longtable}[]{@{}lll@{}}
\toprule
Type & Effect size & CI?\tabularnewline
\midrule
\endhead
Pearson's chi-squared test & Cramér's \emph{V} & \textcolor{ForestGreen}{Yes}\tabularnewline
McNemar's test & Cohen's \emph{g} & \textcolor{ForestGreen}{Yes}\tabularnewline
Goodness of fit & Cramér's \emph{V} & \textcolor{ForestGreen}{Yes}\tabularnewline
\bottomrule
\end{longtable}

For more, see the \texttt{ggpiestats} vignette:
\url{https://indrajeetpatil.github.io/ggstatsplot/articles/web_only/ggpiestats.html}

\hypertarget{ggbarstats}{%
\subsection{\texorpdfstring{\texttt{ggbarstats}}{ggbarstats}}\label{ggbarstats}}

In case you are not a fan of pie charts (for very good reasons), you can
alternatively use \texttt{ggbarstats} function which has a similar syntax-

\begin{Shaded}
\begin{Highlighting}[]
\CommentTok{# for reproducibility}
\KeywordTok{set.seed}\NormalTok{(}\DecValTok{123}\NormalTok{)}

\CommentTok{# plot}
\NormalTok{ggstatsplot}\OperatorTok{::}\KeywordTok{ggbarstats}\NormalTok{(}
  \DataTypeTok{data =}\NormalTok{ ggstatsplot}\OperatorTok{::}\NormalTok{movies_long,}
  \DataTypeTok{x =}\NormalTok{ mpaa,}
  \DataTypeTok{y =}\NormalTok{ genre,}
  \DataTypeTok{sampling.plan =} \StringTok{"jointMulti"}\NormalTok{,}
  \DataTypeTok{title =} \StringTok{"MPAA Ratings by Genre"}\NormalTok{,}
  \DataTypeTok{xlab =} \StringTok{"movie genre"}\NormalTok{,}
  \DataTypeTok{perc.k =} \DecValTok{1}\NormalTok{,}
  \DataTypeTok{x.axis.orientation =} \StringTok{"slant"}\NormalTok{,}
  \DataTypeTok{ggtheme =}\NormalTok{ hrbrthemes}\OperatorTok{::}\KeywordTok{theme_modern_rc}\NormalTok{(),}
  \DataTypeTok{ggstatsplot.layer =} \OtherTok{FALSE}\NormalTok{,}
  \DataTypeTok{ggplot.component =}\NormalTok{ ggplot2}\OperatorTok{::}\KeywordTok{theme}\NormalTok{(}\DataTypeTok{axis.text.x =}\NormalTok{ ggplot2}\OperatorTok{::}\KeywordTok{element_text}\NormalTok{(}\DataTypeTok{face =} \StringTok{"italic"}\NormalTok{)),}
  \DataTypeTok{palette =} \StringTok{"Set2"}\NormalTok{,}
  \DataTypeTok{messages =} \OtherTok{FALSE}
\NormalTok{)}
\end{Highlighting}
\end{Shaded}

\includegraphics[width=1\linewidth]{./figures/paper-ggbarstats1-1}

And, needless to say, there is also a \texttt{grouped\_} variant of this function-

\begin{Shaded}
\begin{Highlighting}[]
\CommentTok{# setup}
\KeywordTok{set.seed}\NormalTok{(}\DecValTok{123}\NormalTok{)}

\CommentTok{# smaller dataset}
\NormalTok{df <-}\StringTok{ }\NormalTok{dplyr}\OperatorTok{::}\KeywordTok{filter}\NormalTok{(}
  \DataTypeTok{.data =}\NormalTok{ forcats}\OperatorTok{::}\NormalTok{gss_cat,}
\NormalTok{  race }\OperatorTok\StringTok{ }\KeywordTok{c}\NormalTok{(}\StringTok{"Black"}\NormalTok{, }\StringTok{"White"}\NormalTok{),}
\NormalTok{  relig }\OperatorTok\StringTok{ }\KeywordTok{c}\NormalTok{(}\StringTok{"Protestant"}\NormalTok{, }\StringTok{"Catholic"}\NormalTok{, }\StringTok{"None"}\NormalTok{),}
  \OperatorTok{!}\NormalTok{partyid }\OperatorTok\StringTok{ }\KeywordTok{c}\NormalTok{(}\StringTok{"No answer"}\NormalTok{, }\StringTok{"Don't know"}\NormalTok{, }\StringTok{"Other party"}\NormalTok{)}
\NormalTok{)}

\CommentTok{# plot}
\NormalTok{ggstatsplot}\OperatorTok{::}\KeywordTok{grouped_ggbarstats}\NormalTok{(}
  \DataTypeTok{data =}\NormalTok{ df,}
  \DataTypeTok{x =}\NormalTok{ relig,}
  \DataTypeTok{y =}\NormalTok{ partyid,}
  \DataTypeTok{grouping.var =}\NormalTok{ race,}
  \DataTypeTok{title.prefix =} \StringTok{"Race"}\NormalTok{,}
  \DataTypeTok{xlab =} \StringTok{"Party affiliation"}\NormalTok{,}
  \DataTypeTok{ggtheme =}\NormalTok{ ggthemes}\OperatorTok{::}\KeywordTok{theme_tufte}\NormalTok{(}\DataTypeTok{base_size =} \DecValTok{12}\NormalTok{),}
  \DataTypeTok{ggstatsplot.layer =} \OtherTok{FALSE}\NormalTok{,}
  \DataTypeTok{messages =} \OtherTok{FALSE}\NormalTok{,}
  \DataTypeTok{title.text =} \StringTok{"Race, religion, and political affiliation"}\NormalTok{,}
  \DataTypeTok{nrow =} \DecValTok{2}
\NormalTok{)}
\end{Highlighting}
\end{Shaded}

\includegraphics[width=1\linewidth]{./figures/paper-ggbarstats2-1}

\hypertarget{summary-of-tests-4}{%
\subsubsection{Summary of tests}\label{summary-of-tests-4}}

This is identical to the \texttt{ggpiestats} function summary of tests.

\hypertarget{gghistostats}{%
\subsection{\texorpdfstring{\texttt{gghistostats}}{gghistostats}}\label{gghistostats}}

In case you would like to see the distribution of a single variable and check if
it is significantly different from a specified value with a one sample test,
this function will let you do that.

\begin{Shaded}
\begin{Highlighting}[]
\CommentTok{# for reproducibility}
\KeywordTok{set.seed}\NormalTok{(}\DecValTok{123}\NormalTok{)}

\CommentTok{# plot}
\NormalTok{ggstatsplot}\OperatorTok{::}\KeywordTok{gghistostats}\NormalTok{(}
  \DataTypeTok{data =}\NormalTok{ iris, }\CommentTok{# dataframe from which variable is to be taken}
  \DataTypeTok{x =}\NormalTok{ Sepal.Length, }\CommentTok{# numeric variable whose distribution is of interest}
  \DataTypeTok{title =} \StringTok{"Distribution of Iris sepal length"}\NormalTok{, }\CommentTok{# title for the plot}
  \DataTypeTok{caption =} \KeywordTok{substitute}\NormalTok{(}\KeywordTok{paste}\NormalTok{(}\KeywordTok{italic}\NormalTok{(}\StringTok{"Source:"}\NormalTok{, }\StringTok{"Ronald Fisher's Iris data set"}\NormalTok{))),}
  \DataTypeTok{type =} \StringTok{"parametric"}\NormalTok{, }\CommentTok{# one sample t-test}
  \DataTypeTok{conf.level =} \FloatTok{0.99}\NormalTok{, }\CommentTok{# changing confidence level for effect size}
  \DataTypeTok{bar.measure =} \StringTok{"mix"}\NormalTok{, }\CommentTok{# what does the bar length denote}
  \DataTypeTok{test.value =} \DecValTok{5}\NormalTok{, }\CommentTok{# default value is 0}
  \DataTypeTok{test.value.line =} \OtherTok{TRUE}\NormalTok{, }\CommentTok{# display a vertical line at test value}
  \DataTypeTok{test.value.color =} \StringTok{"#0072B2"}\NormalTok{, }\CommentTok{# color for the line for test value}
  \DataTypeTok{centrality.para =} \StringTok{"mean"}\NormalTok{, }\CommentTok{# which measure of central tendency is to be plotted}
  \DataTypeTok{centrality.color =} \StringTok{"darkred"}\NormalTok{, }\CommentTok{# decides color for central tendency line}
  \DataTypeTok{binwidth =} \FloatTok{0.10}\NormalTok{, }\CommentTok{# binwidth value (experiment)}
  \DataTypeTok{bf.prior =} \FloatTok{0.8}\NormalTok{, }\CommentTok{# prior width for computing bayes factor}
  \DataTypeTok{messages =} \OtherTok{FALSE}\NormalTok{, }\CommentTok{# turn off the messages}
  \DataTypeTok{ggtheme =}\NormalTok{ hrbrthemes}\OperatorTok{::}\KeywordTok{theme_ipsum_tw}\NormalTok{(), }\CommentTok{# choosing a different theme}
  \DataTypeTok{ggstatsplot.layer =} \OtherTok{FALSE} \CommentTok{# turn off ggstatsplot theme layer}
\NormalTok{)}
\end{Highlighting}
\end{Shaded}

\includegraphics[width=1\linewidth]{./figures/paper-gghistostats1-1}

Additionally, there is also a \texttt{grouped\_} variant of this function that makes it
easy to repeat the same operation across a \textbf{single} grouping variable:

\begin{Shaded}
\begin{Highlighting}[]
\CommentTok{# for reproducibility}
\KeywordTok{set.seed}\NormalTok{(}\DecValTok{123}\NormalTok{)}

\CommentTok{# plot}
\NormalTok{ggstatsplot}\OperatorTok{::}\KeywordTok{grouped_gghistostats}\NormalTok{(}
  \DataTypeTok{data =}\NormalTok{ dplyr}\OperatorTok{::}\KeywordTok{filter}\NormalTok{(}
    \DataTypeTok{.data =}\NormalTok{ ggstatsplot}\OperatorTok{::}\NormalTok{movies_long,}
\NormalTok{    genre }\OperatorTok\StringTok{ }\KeywordTok{c}\NormalTok{(}\StringTok{"Action"}\NormalTok{, }\StringTok{"Action Comedy"}\NormalTok{, }\StringTok{"Action Drama"}\NormalTok{, }\StringTok{"Comedy"}\NormalTok{)}
\NormalTok{  ),}
  \DataTypeTok{x =}\NormalTok{ budget,}
  \DataTypeTok{xlab =} \StringTok{"Movies budget (in million US$)"}\NormalTok{,}
  \DataTypeTok{type =} \StringTok{"robust"}\NormalTok{, }\CommentTok{# use robust location measure}
  \DataTypeTok{grouping.var =}\NormalTok{ genre, }\CommentTok{# grouping variable}
  \DataTypeTok{normal.curve =} \OtherTok{TRUE}\NormalTok{, }\CommentTok{# superimpose a normal distribution curve}
  \DataTypeTok{normal.curve.color =} \StringTok{"red"}\NormalTok{,}
  \DataTypeTok{title.prefix =} \StringTok{"Movie genre"}\NormalTok{,}
  \DataTypeTok{ggtheme =}\NormalTok{ ggthemes}\OperatorTok{::}\KeywordTok{theme_tufte}\NormalTok{(),}
  \DataTypeTok{ggplot.component =} \KeywordTok{list}\NormalTok{( }\CommentTok{# modify the defaults from `ggstatsplot` for each plot}
\NormalTok{    ggplot2}\OperatorTok{::}\KeywordTok{scale_x_continuous}\NormalTok{(}\DataTypeTok{breaks =} \KeywordTok{seq}\NormalTok{(}\DecValTok{0}\NormalTok{, }\DecValTok{200}\NormalTok{, }\DecValTok{50}\NormalTok{), }\DataTypeTok{limits =}\NormalTok{ (}\KeywordTok{c}\NormalTok{(}\DecValTok{0}\NormalTok{, }\DecValTok{200}\NormalTok{)))}
\NormalTok{  ),}
  \DataTypeTok{messages =} \OtherTok{FALSE}\NormalTok{,}
  \DataTypeTok{nrow =} \DecValTok{2}\NormalTok{,}
  \DataTypeTok{title.text =} \StringTok{"Movies budgets for different genres"}
\NormalTok{)}
\end{Highlighting}
\end{Shaded}

\includegraphics[width=1\linewidth]{./figures/paper-gghistostats4-1}

\hypertarget{summary-of-tests-5}{%
\subsubsection{Summary of tests}\label{summary-of-tests-5}}

Following tests are carried out for each type of analyses-

\begin{longtable}[]{@{}ll@{}}
\toprule
Type & Test\tabularnewline
\midrule
\endhead
Parametric & One-sample Student's \emph{t}-test\tabularnewline
Non-parametric & One-sample Wilcoxon test\tabularnewline
Robust & One-sample percentile bootstrap\tabularnewline
Bayes Factor & One-sample Student's \emph{t}-test\tabularnewline
\bottomrule
\end{longtable}

Following effect sizes (and confidence intervals/CI) are available for each type
of test-

\begin{longtable}[]{@{}lll@{}}
\toprule
\begin{minipage}[b]{0.26\columnwidth}\raggedright
Type\strut
\end{minipage} & \begin{minipage}[b]{0.59\columnwidth}\raggedright
Effect size\strut
\end{minipage} & \begin{minipage}[b]{0.07\columnwidth}\raggedright
CI?\strut
\end{minipage}\tabularnewline
\midrule
\endhead
\begin{minipage}[t]{0.26\columnwidth}\raggedright
Parametric\strut
\end{minipage} & \begin{minipage}[t]{0.59\columnwidth}\raggedright
Cohen's \emph{d}, Hedge's \emph{g} (central-and noncentral-\emph{t} distribution based)\strut
\end{minipage} & \begin{minipage}[t]{0.07\columnwidth}\raggedright
\textcolor{ForestGreen}{Yes}\strut
\end{minipage}\tabularnewline
\begin{minipage}[t]{0.26\columnwidth}\raggedright
Non-parametric\strut
\end{minipage} & \begin{minipage}[t]{0.59\columnwidth}\raggedright
\emph{r} (computed as \(Z/\sqrt{N}\))\strut
\end{minipage} & \begin{minipage}[t]{0.07\columnwidth}\raggedright
\textcolor{ForestGreen}{Yes}\strut
\end{minipage}\tabularnewline
\begin{minipage}[t]{0.26\columnwidth}\raggedright
Robust\strut
\end{minipage} & \begin{minipage}[t]{0.59\columnwidth}\raggedright
\(M_{robust}\) (Robust location measure)\strut
\end{minipage} & \begin{minipage}[t]{0.07\columnwidth}\raggedright
\textcolor{ForestGreen}{Yes}\strut
\end{minipage}\tabularnewline
\begin{minipage}[t]{0.26\columnwidth}\raggedright
Bayes Factor\strut
\end{minipage} & \begin{minipage}[t]{0.59\columnwidth}\raggedright
\textcolor{red}{No}\strut
\end{minipage} & \begin{minipage}[t]{0.07\columnwidth}\raggedright
\textcolor{red}{No}\strut
\end{minipage}\tabularnewline
\bottomrule
\end{longtable}

For more, including information about the variant of this function
\texttt{grouped\_gghistostats}, see the \texttt{gghistostats} vignette:
\url{https://indrajeetpatil.github.io/ggstatsplot/articles/web_only/gghistostats.html}

\hypertarget{ggdotplotstats}{%
\subsection{\texorpdfstring{\texttt{ggdotplotstats}}{ggdotplotstats}}\label{ggdotplotstats}}

This function is similar to \texttt{gghistostats}, but is intended to be used when the
numeric variable also has a label.

\begin{Shaded}
\begin{Highlighting}[]
\CommentTok{# for reproducibility}
\KeywordTok{set.seed}\NormalTok{(}\DecValTok{123}\NormalTok{)}

\CommentTok{# plot}
\KeywordTok{ggdotplotstats}\NormalTok{(}
  \DataTypeTok{data =}\NormalTok{ dplyr}\OperatorTok{::}\KeywordTok{filter}\NormalTok{(}\DataTypeTok{.data =}\NormalTok{ gapminder}\OperatorTok{::}\NormalTok{gapminder, continent }\OperatorTok{==}\StringTok{ "Asia"}\NormalTok{),}
  \DataTypeTok{y =}\NormalTok{ country,}
  \DataTypeTok{x =}\NormalTok{ lifeExp,}
  \DataTypeTok{test.value =} \DecValTok{55}\NormalTok{,}
  \DataTypeTok{test.value.line =} \OtherTok{TRUE}\NormalTok{,}
  \DataTypeTok{test.line.labeller =} \OtherTok{TRUE}\NormalTok{,}
  \DataTypeTok{test.value.color =} \StringTok{"red"}\NormalTok{,}
  \DataTypeTok{centrality.para =} \StringTok{"median"}\NormalTok{,}
  \DataTypeTok{centrality.k =} \DecValTok{0}\NormalTok{,}
  \DataTypeTok{title =} \StringTok{"Distribution of life expectancy in Asian continent"}\NormalTok{,}
  \DataTypeTok{xlab =} \StringTok{"Life expectancy"}\NormalTok{,}
  \DataTypeTok{messages =} \OtherTok{FALSE}\NormalTok{,}
  \DataTypeTok{caption =} \KeywordTok{substitute}\NormalTok{(}
    \KeywordTok{paste}\NormalTok{(}
      \KeywordTok{italic}\NormalTok{(}\StringTok{"Source"}\NormalTok{),}
      \StringTok{": Gapminder dataset from https://www.gapminder.org/"}
\NormalTok{    )}
\NormalTok{  )}
\NormalTok{)}
\end{Highlighting}
\end{Shaded}

\includegraphics[width=1\linewidth]{./figures/paper-ggdotplotstats1-1}

As with the rest of the functions in this package, there is also a \texttt{grouped\_}
variant of this function to facilitate looping the same operation for all levels
of a single grouping variable.

\begin{Shaded}
\begin{Highlighting}[]
\CommentTok{# for reproducibility}
\KeywordTok{set.seed}\NormalTok{(}\DecValTok{123}\NormalTok{)}

\CommentTok{# removing factor level with very few no. of observations}
\NormalTok{df <-}\StringTok{ }\NormalTok{dplyr}\OperatorTok{::}\KeywordTok{filter}\NormalTok{(}\DataTypeTok{.data =}\NormalTok{ ggplot2}\OperatorTok{::}\NormalTok{mpg, cyl }\OperatorTok\StringTok{ }\KeywordTok{c}\NormalTok{(}\StringTok{"4"}\NormalTok{, }\StringTok{"6"}\NormalTok{))}

\CommentTok{# plot}
\NormalTok{ggstatsplot}\OperatorTok{::}\KeywordTok{grouped_ggdotplotstats}\NormalTok{(}
  \DataTypeTok{data =}\NormalTok{ df,}
  \DataTypeTok{x =}\NormalTok{ cty,}
  \DataTypeTok{y =}\NormalTok{ manufacturer,}
  \DataTypeTok{xlab =} \StringTok{"city miles per gallon"}\NormalTok{,}
  \DataTypeTok{ylab =} \StringTok{"car manufacturer"}\NormalTok{,}
  \DataTypeTok{type =} \StringTok{"nonparametric"}\NormalTok{, }\CommentTok{# non-parametric test}
  \DataTypeTok{grouping.var =}\NormalTok{ cyl, }\CommentTok{# grouping variable}
  \DataTypeTok{test.value =} \FloatTok{15.5}\NormalTok{,}
  \DataTypeTok{title.prefix =} \StringTok{"cylinder count"}\NormalTok{,}
  \DataTypeTok{point.color =} \StringTok{"red"}\NormalTok{,}
  \DataTypeTok{point.size =} \DecValTok{5}\NormalTok{,}
  \DataTypeTok{point.shape =} \DecValTok{13}\NormalTok{,}
  \DataTypeTok{test.value.line =} \OtherTok{TRUE}\NormalTok{,}
  \DataTypeTok{ggtheme =}\NormalTok{ ggthemes}\OperatorTok{::}\KeywordTok{theme_par}\NormalTok{(),}
  \DataTypeTok{messages =} \OtherTok{FALSE}\NormalTok{,}
  \DataTypeTok{title.text =} \StringTok{"Fuel economy data"}
\NormalTok{)}
\end{Highlighting}
\end{Shaded}

\includegraphics[width=1\linewidth]{./figures/paper-ggdotplotstats2-1}

\hypertarget{summary-of-tests-6}{%
\subsubsection{Summary of tests}\label{summary-of-tests-6}}

This is identical to summary of tests for \texttt{gghistostats}.

\hypertarget{ggcorrmat}{%
\subsection{ggcorrmat}\label{ggcorrmat}}

\texttt{ggcorrmat} makes a correlalogram (a matrix of correlation coefficients) with
minimal amount of code. Just sticking to the defaults itself produces
publication-ready correlation matrices. But, for the sake of exploring the
available options, let's change some of the defaults. For example, multiple
aesthetics-related arguments can be modified to change the appearance of the
correlation matrix.

\begin{Shaded}
\begin{Highlighting}[]
\CommentTok{# for reproducibility}
\KeywordTok{set.seed}\NormalTok{(}\DecValTok{123}\NormalTok{)}

\CommentTok{# as a default this function outputs a correlalogram plot}
\NormalTok{ggstatsplot}\OperatorTok{::}\KeywordTok{ggcorrmat}\NormalTok{(}
  \DataTypeTok{data =}\NormalTok{ ggplot2}\OperatorTok{::}\NormalTok{msleep,}
  \DataTypeTok{corr.method =} \StringTok{"robust"}\NormalTok{, }\CommentTok{# correlation method}
  \DataTypeTok{sig.level =} \FloatTok{0.001}\NormalTok{, }\CommentTok{# threshold of significance}
  \DataTypeTok{p.adjust.method =} \StringTok{"holm"}\NormalTok{, }\CommentTok{# p-value adjustment method for multiple comparisons}
  \DataTypeTok{cor.vars =} \KeywordTok{c}\NormalTok{(sleep_rem, awake}\OperatorTok{:}\NormalTok{bodywt), }\CommentTok{# a range of variables can be selected}
  \DataTypeTok{cor.vars.names =} \KeywordTok{c}\NormalTok{(}
    \StringTok{"REM sleep"}\NormalTok{, }\CommentTok{# variable names}
    \StringTok{"time awake"}\NormalTok{,}
    \StringTok{"brain weight"}\NormalTok{,}
    \StringTok{"body weight"}
\NormalTok{  ),}
  \DataTypeTok{matrix.type =} \StringTok{"upper"}\NormalTok{, }\CommentTok{# type of visualization matrix}
  \DataTypeTok{colors =} \KeywordTok{c}\NormalTok{(}\StringTok{"#B2182B"}\NormalTok{, }\StringTok{"white"}\NormalTok{, }\StringTok{"#4D4D4D"}\NormalTok{),}
  \DataTypeTok{title =} \StringTok{"Correlalogram for mammals sleep dataset"}\NormalTok{,}
  \DataTypeTok{subtitle =} \StringTok{"sleep units: hours; weight units: kilograms"}
\NormalTok{)}
\end{Highlighting}
\end{Shaded}

\includegraphics[width=1\linewidth]{./figures/paper-ggcorrmat1-1}

Note that if there are \texttt{NA}s present in the selected variables, the legend will
display minimum, median, and maximum number of pairs used for correlation tests.

There is a \texttt{grouped\_} variant of this function that makes it
easy to repeat the same operation across a \textbf{single} grouping variable:

\begin{Shaded}
\begin{Highlighting}[]
\CommentTok{# for reproducibility}
\KeywordTok{set.seed}\NormalTok{(}\DecValTok{123}\NormalTok{)}

\CommentTok{# plot}
\CommentTok{# let's use only 50% of the data to speed up the process}
\NormalTok{ggstatsplot}\OperatorTok{::}\KeywordTok{grouped_ggcorrmat}\NormalTok{(}
  \DataTypeTok{data =}\NormalTok{ dplyr}\OperatorTok{::}\KeywordTok{filter}\NormalTok{(}
    \DataTypeTok{.data =}\NormalTok{ ggstatsplot}\OperatorTok{::}\NormalTok{movies_long,}
\NormalTok{    genre }\OperatorTok\StringTok{ }\KeywordTok{c}\NormalTok{(}\StringTok{"Action"}\NormalTok{, }\StringTok{"Action Comedy"}\NormalTok{, }\StringTok{"Action Drama"}\NormalTok{, }\StringTok{"Comedy"}\NormalTok{)}
\NormalTok{  ),}
  \DataTypeTok{cor.vars =}\NormalTok{ length}\OperatorTok{:}\NormalTok{votes,}
  \DataTypeTok{corr.method =} \StringTok{"np"}\NormalTok{,}
  \DataTypeTok{colors =} \KeywordTok{c}\NormalTok{(}\StringTok{"#cbac43"}\NormalTok{, }\StringTok{"white"}\NormalTok{, }\StringTok{"#550000"}\NormalTok{),}
  \DataTypeTok{grouping.var =}\NormalTok{ genre, }\CommentTok{# grouping variable}
  \DataTypeTok{digits =} \DecValTok{3}\NormalTok{, }\CommentTok{# number of digits after decimal point}
  \DataTypeTok{title.prefix =} \StringTok{"Movie genre"}\NormalTok{,}
  \DataTypeTok{messages =} \OtherTok{FALSE}\NormalTok{,}
  \DataTypeTok{nrow =} \DecValTok{2}
\NormalTok{)}
\end{Highlighting}
\end{Shaded}

\includegraphics[width=1\linewidth]{./figures/paper-ggcorrmat3-1}

\hypertarget{summary-of-tests-7}{%
\subsubsection{Summary of tests}\label{summary-of-tests-7}}

Following tests are carried out for each type of analyses. Additionally, the
correlation coefficients (and their confidence intervals) are used as effect
sizes-

\begin{longtable}[]{@{}lll@{}}
\toprule
Type & Test & CI?\tabularnewline
\midrule
\endhead
Parametric & Pearson's correlation coefficient & \textcolor{ForestGreen}{Yes}\tabularnewline
Non-parametric & Spearman's rank correlation coefficient & \textcolor{ForestGreen}{Yes}\tabularnewline
Robust & Percentage bend correlation coefficient & \textcolor{red}{No}\tabularnewline
Bayes Factor & Pearson's correlation coefficient & \textcolor{red}{No}\tabularnewline
\bottomrule
\end{longtable}

For examples and more information, see the \texttt{ggcorrmat} vignette:
\url{https://indrajeetpatil.github.io/ggstatsplot/articles/web_only/ggcorrmat.html}

\hypertarget{ggcoefstats} is the default). The estimate can either be
  effect sizes (for tests that depend on the \texttt{F} statistic) or regression
  coefficients (for tests with \texttt{t} and \texttt{z} statistic), etc. The function will, by
  default, display a helpful \texttt{x}-axis label that should clear up what estimates are
  being displayed. The confidence intervals can sometimes be asymmetric if
  bootstrapping was used.
\item
  The caption will always contain diagnostic information, if available, about
  models that can be useful for model selection: The smaller the Akaike's
  Information Criterion (\textbf{AIC}) and the Bayesian Information Criterion (\textbf{BIC})
  values, the ``better'' the model is. Additionally, the higher the
  \textbf{log-likelihood} value the ``better'' is the model fit.
\item
  The output of this function will be a \texttt{ggplot2} object and, thus, it can be
  further modified (e.g., change themes, etc.) with \texttt{ggplot2} functions.
\end{itemize}

\begin{Shaded}
\begin{Highlighting}[]
\CommentTok{# for reproducibility}
\KeywordTok{set.seed}\NormalTok{(}\DecValTok{123}\NormalTok{)}

\CommentTok{# model}
\NormalTok{mod <-}\StringTok{ }\NormalTok{stats}\OperatorTok{::}\KeywordTok{lm}\NormalTok{(}\DataTypeTok{formula =}\NormalTok{ mpg }\OperatorTok{~}\StringTok{ }\NormalTok{am }\OperatorTok{*}\StringTok{ }\NormalTok{cyl, }\DataTypeTok{data =}\NormalTok{ mtcars)}

\CommentTok{# plot}
\NormalTok{ggstatsplot}\OperatorTok{::}\KeywordTok{ggcoefstats}\NormalTok{(}\DataTypeTok{x =}\NormalTok{ mod, }\DataTypeTok{title =} \StringTok{"linear model"}\NormalTok{)}
\end{Highlighting}
\end{Shaded}

\includegraphics[width=1\linewidth]{./figures/paper-ggcoefstats1-1}

This default plot can be further modified to one's liking with additional arguments
(also, let's use a robust linear model instead of a simple linear model now):

\begin{Shaded}
\begin{Highlighting}[]
\CommentTok{# for reproducibility}
\KeywordTok{set.seed}\NormalTok{(}\DecValTok{123}\NormalTok{)}

\CommentTok{# model}
\NormalTok{mod <-}\StringTok{ }\NormalTok{MASS}\OperatorTok{::}\KeywordTok{rlm}\NormalTok{(}\DataTypeTok{formula =}\NormalTok{ mpg }\OperatorTok{~}\StringTok{ }\NormalTok{am }\OperatorTok{*}\StringTok{ }\NormalTok{cyl, }\DataTypeTok{data =}\NormalTok{ mtcars)}

\CommentTok{# plot}
\NormalTok{ggstatsplot}\OperatorTok{::}\KeywordTok{ggcoefstats}\NormalTok{(}
  \DataTypeTok{x =}\NormalTok{ mod,}
  \DataTypeTok{point.color =} \StringTok{"red"}\NormalTok{,}
  \DataTypeTok{point.shape =} \DecValTok{15}\NormalTok{,}
  \DataTypeTok{vline.color =} \StringTok{"#CC79A7"}\NormalTok{,}
  \DataTypeTok{vline.linetype =} \StringTok{"dotdash"}\NormalTok{,}
  \DataTypeTok{stats.label.size =} \FloatTok{3.5}\NormalTok{,}
  \DataTypeTok{stats.label.color =} \KeywordTok{c}\NormalTok{(}\StringTok{"#0072B2"}\NormalTok{, }\StringTok{"#D55E00"}\NormalTok{, }\StringTok{"darkgreen"}\NormalTok{),}
  \DataTypeTok{title =} \StringTok{"Car performance predicted by transmission & cylinder count"}\NormalTok{,}
  \DataTypeTok{subtitle =} \StringTok{"Source: 1974 Motor Trend US magazine"}\NormalTok{,}
  \DataTypeTok{ggtheme =}\NormalTok{ hrbrthemes}\OperatorTok{::}\KeywordTok{theme_ipsum_ps}\NormalTok{(),}
  \DataTypeTok{ggstatsplot.layer =} \OtherTok{FALSE}
\NormalTok{) }\OperatorTok{+}\StringTok{ }\CommentTok{# further modification with the ggplot2 commands}
\StringTok{  }\NormalTok{ggplot2}\OperatorTok{::}\KeywordTok{scale_y_discrete}\NormalTok{(}\DataTypeTok{labels =} \KeywordTok{c}\NormalTok{(}\StringTok{"transmission"}\NormalTok{, }\StringTok{"cylinders"}\NormalTok{, }\StringTok{"interaction"}\NormalTok{)) }\OperatorTok{+}
\StringTok{  }\NormalTok{ggplot2}\OperatorTok{::}\KeywordTok{labs}\NormalTok{(}\DataTypeTok{x =} \StringTok{"regression coefficient"}\NormalTok{, }\DataTypeTok{y =} \OtherTok{NULL}\NormalTok{)}
\end{Highlighting}
\end{Shaded}

\includegraphics[width=1\linewidth]{./figures/paper-ggcoefstats2-1}

Most of the regression models that are supported in the \texttt{broom} and
\texttt{broom.mixed} packages with \texttt{tidy} and \texttt{glance} methods are also supported by
\texttt{ggcoefstats}. For example-

\texttt{aareg}, \texttt{anova}, \texttt{aov}, \texttt{aovlist}, \texttt{Arima}, \texttt{bigglm}, \texttt{biglm}, \texttt{brmsfit},
\texttt{btergm}, \texttt{cch}, \texttt{clm}, \texttt{clmm}, \texttt{confusionMatrix}, \texttt{coxph}, \texttt{drc}, \texttt{emmGrid},
\texttt{epi.2by2}, \texttt{ergm}, \texttt{felm}, \texttt{fitdistr}, \texttt{glmerMod}, \texttt{glmmTMB}, \texttt{gls}, \texttt{gam},
\texttt{Gam}, \texttt{gamlss}, \texttt{garch}, \texttt{glm}, \texttt{glmmadmb}, \texttt{glmmPQL}, \texttt{glmmTMB}, \texttt{glmRob},
\texttt{glmrob}, \texttt{gmm}, \texttt{ivreg}, \texttt{lm}, \texttt{lm.beta}, \texttt{lmerMod}, \texttt{lmodel2}, \texttt{lmRob},
\texttt{lmrob}, \texttt{mcmc}, \texttt{MCMCglmm}, \texttt{mclogit}, \texttt{mmclogit}, \texttt{mediate}, \texttt{mjoint}, \texttt{mle2},
\texttt{mlm}, \texttt{multinom}, \texttt{negbin}, \texttt{nlmerMod}, \texttt{nlrq}, \texttt{nls}, \texttt{orcutt}, \texttt{plm}, \texttt{polr},
\texttt{ridgelm}, \texttt{rjags}, \texttt{rlm}, \texttt{rlmerMod}, \texttt{rq}, \texttt{speedglm}, \texttt{speedlm}, \texttt{stanreg},
\texttt{survreg}, \texttt{svyglm}, \texttt{svyolr}, \texttt{svyglm}, etc.

Although not shown here, this function can also be used to carry out both
frequentist and Bayesian random-effects meta-analysis.

For a more exhaustive account of this function, see the associated vignette-
\url{https://indrajeetpatil.github.io/ggstatsplot/articles/web_only/ggcoefstats.html}

\hypertarget{working-with-custom-plots}{%
\subsection{Working with custom plots}\label{working-with-custom-plots}}

Sometimes you may not like the defaults in a plot produced by \texttt{ggstatsplot}. In
such cases, you can use other custom plots (from \texttt{ggplot2} or other graphics
packages in \texttt{R}) and still use \texttt{ggstatsplot} functions to display results from relevant
statistical test.

For example, in the following chunk, we will create plot (\emph{pirateplot}) using
\texttt{yarrr} package and use \texttt{ggstatsplot} function for extracting results.

\begin{Shaded}
\begin{Highlighting}[]
\CommentTok{# for reproducibility}
\KeywordTok{set.seed}\NormalTok{(}\DecValTok{123}\NormalTok{)}

\CommentTok{# loading the needed libraries}
\KeywordTok{library}\NormalTok{(yarrr)}
\KeywordTok{library}\NormalTok{(ggstatsplot)}

\CommentTok{# using `ggstatsplot` to get call with statistical results}
\NormalTok{stats_results <-}
\StringTok{  }\NormalTok{ggstatsplot}\OperatorTok{::}\KeywordTok{ggbetweenstats}\NormalTok{(}
    \DataTypeTok{data =}\NormalTok{ ChickWeight,}
    \DataTypeTok{x =}\NormalTok{ Time,}
    \DataTypeTok{y =}\NormalTok{ weight,}
    \DataTypeTok{return =} \StringTok{"subtitle"}\NormalTok{,}
    \DataTypeTok{messages =} \OtherTok{FALSE}
\NormalTok{  )}

\CommentTok{# using `yarrr` to create plot}
\NormalTok{yarrr}\OperatorTok{::}\KeywordTok{pirateplot}\NormalTok{(}
  \DataTypeTok{formula =}\NormalTok{ weight }\OperatorTok{~}\StringTok{ }\NormalTok{Time,}
  \DataTypeTok{data =}\NormalTok{ ChickWeight,}
  \DataTypeTok{theme =} \DecValTok{1}\NormalTok{,}
  \DataTypeTok{main =}\NormalTok{ stats_results}
\NormalTok{)}
\end{Highlighting}
\end{Shaded}

\includegraphics[width=1\linewidth]{./figures/paper-pirateplot-1}

\hypertarget{overall-consistency-in-api}{%
\subsection{Overall consistency in API}\label{overall-consistency-in-api}}

Attempt has been made to make the application program interface (API) consistent
enough that no struggle is expected while thinking about specifying function
calls-

\begin{itemize}
\tightlist
\item
  When a given function depends on variables in a dataframe, \texttt{data} argument
  must always be specified.
\item
  Often, package functions relevant for between-subjects versus
  within-subjects design expect tidy/long versus Cartesian/wide format data,
  respectively. \texttt{ggstatsplot} functions consistently expect tidy/long form
  data.
\item
  All functions accept both quoted (\texttt{x\ =\ "var1"}) and unquoted (\texttt{x\ =\ var1})
  arguments.
\end{itemize}

\begin{figure}
\centering
\includegraphics{figures/arguments.png}
\caption{Consistent API}
\end{figure}

These set principles combined with the fact that almost all functions produce
publication-ready plots that require very few arguments if one finds the
aesthetic and statistical defaults satisfying make the syntax much less
cognitively demanding and easy to remember/reconstruct.

\hypertarget{helpful-messages-and-aesthetic-modifications}{%
\subsection{Helpful messages and aesthetic modifications}\label{helpful-messages-and-aesthetic-modifications}}

\texttt{ggstatsplot} is a very chatty package and will by default print helpful notes
on assumptions about statistical tests, warnings, etc. If you don't want your
console to be cluttered with such messages, they can be turned off by setting
argument \texttt{messages\ =\ FALSE} in the function call.

All relevant functions in \texttt{ggstatsplot} have a \texttt{return} argument which can be
used to not only return plots (which is the default), but also to return a
\texttt{subtitle} or \texttt{caption}, which are objects of type \texttt{call} and can be used to
display statistical details in conjunction with a custom plot and at a custom
location in the plot.

Additionally, all functions share the \texttt{ggtheme} and \texttt{palette} arguments that can
be used to specify your favorite \texttt{ggplot} theme and color palette.

\newpage

\hypertarget{appendix}{%
\section{Appendix}\label{appendix}}

\hypertarget{appendix-a-documentation}{%
\subsection{Appendix A: Documentation}\label{appendix-a-documentation}}

There are three main documents one can rely on to learn how to use
\texttt{ggstatsplot}:

\begin{itemize}
\item
  \textbf{Presentation}:
  The quickest (and the most fun) way to get an overview of
  the philosophy behind this package and the offered functionality is to go
  through the following slides:
  \url{https://indrajeetpatil.github.io/ggstatsplot_slides/slides/ggstatsplot_presentation.html\#1}
\item
  \textbf{Manual}:\\
  The \texttt{CRAN} reference manual provides detailed documentation about arguments
  for each function and examples:
  \url{https://cran.r-project.org/web/packages/ggstatsplot/ggstatsplot.pdf}
\item
  \textbf{Vignettes}:\\
  Vignettes contain probably the most detailed exposition. Every single
  function in \texttt{ggstatsplot} has an associated vignette which describes in
  depth how to use the function and modify the defaults to customize the plot
  to your liking. All these vignettes can be accessed from the package
  website: \url{https://indrajeetpatil.github.io/ggstatsplot/articles/}
\end{itemize}

\hypertarget{appendix-b-suggestions}{%
\subsection{Appendix B: Suggestions}\label{appendix-b-suggestions}}

If you find any bugs or have any suggestions/remarks, please file an issue on
\texttt{GitHub} repository for this package:
\url{https://github.com/IndrajeetPatil/ggstatsplot/issues}

\hypertarget{appendix-c-contributing}{%
\subsection{Appendix C: Contributing}\label{appendix-c-contributing}}

\texttt{ggstatsplot} is happy to receive bug reports, suggestions, questions, and (most
of all) contributions to fix problems and add features. Pull Requests for
contributions are encouraged.

Here are some simple ways in which one can contribute (in the increasing order
of commitment):

\begin{itemize}
\item
  Read and correct any inconsistencies in the
  \href{https://indrajeetpatil.github.io/ggstatsplot/}{documentation}
\item
  Raise issues about bugs or wanted features
\item
  Review code
\item
  Add new functionality (in the form of new plotting functions or helpers for
  preparing subtitles)
\end{itemize}

Please note that this project is released with a
\href{https://github.com/IndrajeetPatil/ggstatsplot/blob/master/CODE_OF_CONDUCT.md}{Contributor Code of Conduct}. By participating in this project you agree to abide by its terms.

\hypertarget{appendix-d-session-information}{%
\subsection{Appendix D: Session information}\label{appendix-d-session-information}}

For reproducibility purposes, the details about the session information in which
this document was rendered, see-
\url{https://indrajeetpatil.github.io/ggstatsplot/articles/web_only/session_info.html}

\newpage

\hypertarget{references}{%
\section*{References}\label{references}}
\addcontentsline{toc}{section}{References}

\hypertarget{refs}{}
\leavevmode\hypertarget{ref-associationPublicationManualAmerican2009}{}%
Association, A. P. (2009). \emph{Publication Manual of the American Psychological Association, 6th Edition}. Washington, DC: American Psychological Association.

\leavevmode\hypertarget{ref-clevelandElementsGraphingData1985}{}%
Cleveland, W. S. (1985). \emph{The Elements of Graphing Data} (1st edition). Monterey, Cal: Wadsworth, Inc.

\leavevmode\hypertarget{ref-lilienfeldFiftyPsychologicalPsychiatric2015}{}%
Lilienfeld, S. O., Sauvign\a'e, K. C., Lynn, S. J., Cautin, R. L., Latzman, R. D., \& Waldman, I. D. (2015). Fifty psychological and psychiatric terms to avoid: A list of inaccurate, misleading, misused, ambiguous, and logically confused words and phrases. \emph{Frontiers in Psychology}, \emph{6}. \url{https://doi.org/10.3389/fpsyg.2015.01100}

\leavevmode\hypertarget{ref-nuijtenPrevalenceStatisticalReporting2016}{}%
Nuijten, M. B., Hartgerink, C. H. J., van Assen, M. A. L. M., Epskamp, S., \& Wicherts, J. M. (2016). The prevalence of statistical reporting errors in psychology (19852013). \emph{Behavior Research Methods}, \emph{48}(4), 1205--1226. \url{https://doi.org/10.3758/s13428-015-0664-2}

\leavevmode\hypertarget{ref-tufteVisualDisplayQuantitative2001}{}%
Tufte, E. R. (2001). \emph{The Visual Display of Quantitative Information} (2nd edition edition). Cheshire, Conn: Graphics Press.

\leavevmode\hypertarget{ref-wickhamTidyData2014}{}%
Wickham, H. (2014). Tidy Data. \emph{Journal of Statistical Software}, \emph{59}(1), 1--23. \url{https://doi.org/10.18637/jss.v059.i10}

\leavevmode\hypertarget{ref-wickhamGgplot2ElegantGraphics2016}{}%
Wickham, H. (2016). \emph{Ggplot2: Elegant Graphics for Data Analysis} (2nd ed. 2016 edition). New York, NY: Springer.


\end{document}
